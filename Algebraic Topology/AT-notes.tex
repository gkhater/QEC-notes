%===============================================
%   rhnotes.tex — Framework for RH-coding Notes
%===============================================
\documentclass[11pt,a4paper]{article}

%----- ENHANCED TYPOGRAPHY -----
\usepackage[utf8]{inputenc}
\usepackage[T1]{fontenc}
\usepackage{lmodern}        % clean vector font
\usepackage{microtype}      % better justification & kerning
\usepackage{palatino} 
\usepackage{braket}    % Palatino for text & math

%----- PAGE LAYOUT -----
\usepackage{geometry}
\geometry{top=1in, bottom=1in, left=1in, right=1in}
\usepackage{setspace}
\onehalfspacing  % 1.5 line spacing

%----- FANCY HEADERS & FOOTERS -----
\usepackage{fancyhdr}
\pagestyle{fancy}
\fancyhf{}
% page number outside, header text inside
\fancyhead[LE]{\small Georges Khater}
\fancyhead[RE]{\small Notes on RH-coding}
\fancyhead[LO]{\small \rightmark}
\fancyhead[RO]{\small \leftmark}
\renewcommand{\headrulewidth}{0.4pt}
\renewcommand{\footrulewidth}{0pt}

% make sections feed into \leftmark/\rightmark
\renewcommand{\sectionmark}[1]{\markboth{#1}{}}
\renewcommand{\subsectionmark}[1]{\markright{#1}}

%----- SECTION NUMBERING & TOC DEPTH -----
\setcounter{secnumdepth}{3}  % number down to \subsubsection
\setcounter{tocdepth}{2}     % show ToC down to \subsection

%----- AMS MATH & THEOREM STYLES -----
\usepackage{amsmath,amssymb,mathtools}
\usepackage{amsthm}

% definitions, examples, remarks upright
\theoremstyle{definition}
\newtheorem{definition}{Definition}[section]
\newtheorem{example}[definition]{Example}
\newtheorem{remark}[definition]{Remark}

% theorems, lemmas, corollaries italic
\theoremstyle{plain}
\newtheorem{theorem}[definition]{Theorem}
\newtheorem{lemma}[definition]{Lemma}
\newtheorem{proposition}[definition]{Proposition}
\newtheorem{corollary}[definition]{Corollary}

% unnumbered proof environment
\theoremstyle{remark}

%----- OTHER PACKAGES -----
\usepackage{graphicx}
\usepackage{tikz}
\usetikzlibrary{calc, matrix, decorations.pathreplacing, positioning}
\usepackage{hyperref}
\hypersetup{colorlinks,
linkcolor=blue, citecolor=purple, urlcolor=teal}
\usepackage{enumitem}
\setlist[itemize]{nosep, left=1.5em}
\usepackage{booktabs}
\usepackage{listings}
\lstset{
basicstyle=\ttfamily\small,
numbers=left,
numbersep=5pt,
frame=single,
breaklines=true
}
\usepackage{xcolor}
\definecolor{shade}{HTML}{F5F5F5}
\usepackage{float}
%----- CUSTOM MACROS -----
\newcommand{\F}{\mathbb{F}}
\newcommand{\code}[1]{\texttt{#1}}
\newcommand{\bsc}{\mathrm{BSC}}
\newcommand{\dist}[2]{d\bigl(#1,#2\bigr)}
\newcommand{\R}{\mathbb{R}}

%----- TITLE METADATA -----
\title{\LARGE\bfseries Algebraic Topology}
\author{Georges Khater \\ \small American University of Beirut}
\date{\today}

%===============================================
\begin{document}
\maketitle
\tableofcontents
\bigskip

\section{Topological Spaces} 

\subsection{Basic Notions} 
%--------- Topology ---------
\begin{definition}[Topology]
Let \(X\) be a set.  A {\it topology} \(\tau\) on \(X\) is a collection \(\tau\subseteq\mathcal P(X)\) such that:
\begin{enumerate}
  \item \(\emptyset\in\tau\) and \(X\in\tau\).
  \item Any union of elements of \(\tau\) lies in \(\tau\).
  \item Any finite intersection of elements of \(\tau\) lies in \(\tau\).
\end{enumerate}
\end{definition}

\noindent\textbf{Example (\(\R\) with the usual metric).}
Let \(d(x,y)=|x-y|\).  Then
\[
  \tau_d \;=\;\{\,U\subseteq\R : (\forall x\in U)\;\exists\varepsilon>0:\;(x-\varepsilon,x+\varepsilon)\subseteq U\}
\]
is a topology on \(\R\).

%--------- Topological Space ---------
\begin{definition}[Topological Space]
A {\it topological space} is a pair \((X,\tau)\) where \(X\) is a set and \(\tau\) is a topology on \(X\).  Elements of \(\tau\) are called {\it open sets}.
\end{definition}

\noindent\textbf{Example.}
\((\R,\tau_d)\) is a topological space.

%--------- Basis ---------
\begin{definition}[Basis]
Let \((X,\mathcal O)\) be a topological space.  A subcollection \(\mathcal B \subseteq \mathcal O\) is called a \emph{basis} for the topology \(\mathcal O\) if for every \(U \in \mathcal O\) there exists a subfamily \(\mathcal B_U \subseteq \mathcal B\) such that
\[
  U \;=\; \bigcup_{B \in \mathcal B_U} B.
\]
\end{definition}

\noindent\textbf{Example (Basis for \(\tau_d\)).}
\[
  \mathcal B \;=\;\{(a,b)\subset\R : a<b\}
\]
is a basis: every open set in \(\tau_d\) is a union of open intervals.

%--------- Subbasis ---------
\begin{definition}[Subbasis]
Let \((X,\mathcal O)\) be a topological space.  A collection \(\mathcal S \subseteq \mathcal O\) is called a \emph{subbasis} for the topology \(\mathcal O\) if the family of all finite intersections of elements of \(\mathcal S\),
\[
  \bigl\{\,S_{1}\cap\cdots\cap S_{n} : n\ge1,\;S_{i}\in\mathcal S\bigr\},
\]
is a basis for \(\mathcal O\).
\end{definition}

\noindent\textbf{Example (Subbasis for \(\tau_d\)).}
\[
  \mathcal S
  = \{\,(-\infty,b)\;:\;b\in\R\}\;\cup\;\{\, (a,\infty)\;:\;a\in\R\}.
\]
Finite intersections give \((a,b)\), and unions of those recover every open set.

\begin{definition}[Continous map] 
  A map $f \cdot X \to Y$ between topological spaces is \emph{continuous} if the pre-image $f^{-1}(V)$ of each open set $V$ of $Y$ is open in $X$. 
  Dually: A map is continuous if the pre-image of each closed set is closed.   
\end{definition}

\noindent\textbf{Example $id(X)$} The identity $id(X) \colon X \to X$ is always continous. 

%--------- Initial (Weak) Topology ---------
\begin{definition}[Initial (Weak) Topology]
Let \(X\) be a set and \(\{\,f_i:X\to Y_i\}_{i\in I}\) a family of maps into topological spaces \((Y_i,\tau_i)\).  
The {\it initial topology} on \(X\) is the coarsest topology making every \(f_i\) continuous.  Equivalently, it is generated by the subbasis
\[
  \{\,f_i^{-1}(U) : i\in I,\;U\in\tau_i\}.
\]
\end{definition}

\begin{definition}[Homeomorphism]
  A \emph{homeomorphism} $f \colon X \to Y$ is a continuous map with a continuous inverse 
  $g \colon Y \to X$. 

  Spaces $X$ and $Y$ are \emph{homeomorphic} if there exists a homeomorphism between them. 
\end{definition}

\begin{definition}[Open and closed maps]
  A map $f \colon X \to Y$ between spaces is \emph{open (closed)} if the image of each open (closed) set is again open (closed).
\end{definition}

We fix a topological space $X$ and a subset $A$. 
\begin{definition}[Closure and Density]
  The intersection of the closed sets which contains $A$ is denoted $\overline{A}$ 
  and called \emph{closure} of $A$ in $X$. 

  A set $A$ is \emph{dense} in $X$ if $\overline{A} = X$. 
\end{definition}

\begin{definition}[Interior and Boundary]
  The \emph{interior} of $A$ is the union of the open sets contained in $A$. 

  The \emph{boundary} of $A$ in $X$ is 
  $$\operatorname{Bd}(A) = \overline{A} \cap \left(\overline{X \setminus A}\right)$$
\end{definition}

% ===== Neighbourhoods =====
\begin{definition}[Open Neighbourhood]
Let \((X,\tau)\) be a topological space and \(A\subseteq X\).  An \emph{open neighbourhood} of \(A\) is any open set \(U\in\tau\) with \(A\subseteq U\).
\end{definition}

\begin{definition}[Neighbourhood]
A set \(B\subseteq X\) is a \emph{neighbourhood} of \(A\subseteq X\) if there exists an open neighbourhood \(U\in\tau\) of \(A\) such that \(U\subseteq B\).
\end{definition}

\begin{definition}[Neighbourhood Basis]
A collection \(\mathcal N_x\) of neighbourhoods of a point \(x\in X\) is a \emph{neighbourhood basis} at \(x\) if for every neighbourhood \(B\) of \(x\) there is some \(N\in\mathcal N_x\) with \(N\subseteq B\).
\end{definition}

% ===== Continuity via Neighbourhoods =====
\begin{definition}[Continuity at a Point]
A map \(f\colon X\to Y\) between topological spaces is \emph{continuous at \(x\in X\)} if for every neighbourhood \(V\) of \(f(x)\) in \(Y\) there exists a neighbourhood \(U\) of \(x\) in \(X\) such that
\[
  f(U)\;\subseteq\;V.
\]
\end{definition}

% ===== Comparing Topologies =====
\begin{definition}[Finer and Coarser Topologies]
Let \(\mathcal O_1,\mathcal O_2\) be two topologies on the same set \(X\).  We say \(\mathcal O_2\) is \emph{finer} than \(\mathcal O_1\) (and \(\mathcal O_1\) is \emph{coarser} than \(\mathcal O_2\)) if and only if the identity map
\[
  \mathrm{id}\colon (X,\mathcal O_2)\;\longrightarrow\;(X,\mathcal O_1)
\]
is continuous.
\end{definition}

% ===== Intersection of Topologies =====
\begin{proposition}[Intersection of Topologies]
If \(\{\mathcal O_j\}_{j\in J}\) is any family of topologies on \(X\), then
\[
  \bigcap_{j\in J}\mathcal O_j
\]
is itself a topology on \(X\).
\end{proposition}

% ===== Separation Axioms =====
\begin{definition}[T\textsubscript{1} Space]
A topological space \((X,\tau)\) is called a \emph{T\textsubscript{1} space} if every singleton \(\{x\}\subseteq X\) is closed.
\end{definition}

\begin{definition}[T\textsubscript{2} / Hausdorff Space]
A space \((X,\tau)\) is \emph{T\textsubscript{2}} (or \emph{Hausdorff}) if for every pair of distinct points \(x,y\in X\) there exist disjoint open neighbourhoods \(U\ni x\) and \(V\ni y\) with \(U\cap V=\varnothing\).
\end{definition}

\begin{definition}[T\textsubscript{3} / Regular Space]
\((X,\tau)\) is \emph{regular} if it is T\textsubscript{1} and whenever \(x\in X\) and \(A\subseteq X\) is closed with \(x\notin A\), there exist disjoint open sets \(U\ni x\) and \(V\supseteq A\) such that \(U\cap V=\varnothing\).
\end{definition}

\begin{definition}[T\textsubscript{4} / Normal Space]
\((X,\tau)\) is \emph{normal} if it is T\textsubscript{1} and whenever \(A,B\subseteq X\) are disjoint closed sets, there exist disjoint open sets \(U\supseteq A\) and \(V\supseteq B\) with \(U\cap V=\varnothing\).
\end{definition}

\begin{definition}[Completely Regular Space]
\((X,\tau)\) is \emph{completely regular} if it is Hausdorff (T\textsubscript{2}) and for every closed set \(A\subseteq X\) and point \(x\notin A\), there exists a continuous map \(f\colon X\to [0,1]\) with \(f(x)=1\) and \(f(A)=\{0\}\).
\end{definition}

% ===== Key Theorems =====
\begin{theorem}[Urysohn Lemma]
Let \(X\) be a normal (T\textsubscript{4}) space, and let \(A,B\subseteq X\) be disjoint closed sets.  Then there exists a continuous function
\[
  f\colon X\;\longrightarrow\;[0,1]
  \quad\text{such that}\quad
  f(A)=\{0\},\quad f(B)=\{1\}.
\]
\end{theorem}

\begin{theorem}[Tietze Extension Theorem]
Let \(X\) be a normal (T\textsubscript{4}) space and \(A\subseteq X\) a closed subset.  Then any continuous map
\[
  f\colon A \;\longrightarrow\;[0,1]
\]
admits a continuous extension
\(\tilde f\colon X\to[0,1]\) with \(\tilde f|_A = f\).
\end{theorem}

% ===== First-Countable =====
\begin{definition}[First-Countable]
A topological space \((X,\tau)\) is said to be \emph{first-countable} if for each point \(x\in X\) there exists a countable collection of neighbourhoods 
\[
  \{\,U_{n}(x)\}_{n=1}^\infty
\]
such that for every neighbourhood \(V\) of \(x\), some \(U_n(x)\subseteq V\).  Equivalently, each point has a countable local basis.
\end{definition}

% ===== Key Points: Metrics ↔ Topology =====
\begin{itemize}
  \item \textbf{Metric induces a topology:}  
    In a metric space \((X,d)\), the open balls 
    \(\;U_\varepsilon(x)=\{\,y:d(x,y)<\varepsilon\}\)
    form a basis for the topology \(\tau_d\).

  \item \textbf{Metrizable spaces:}  
    A space \((X,\tau)\) is \emph{metrizable} if \(\tau=\tau_d\) for some metric \(d\).  
    Every metrizable space is first-countable (e.g.\ use the countable balls \(U_{1/n}(x)\)).

  \item \textbf{Properties of Metric spaces:} Metric spaces respect all sepration properties $T_1, \cdots, T_4$. 
\end{itemize}

\begin{definition}[Directed Set]
A \emph{directed set} is a pair \((I,\le)\) where \(\le\) is a reflexive, transitive relation on \(I\) such that
\[
  \forall\,i,j\in I\;\;\exists\,k\in I:\;i\le k\ \text{and}\ j\le k.
\]
\end{definition}

\begin{definition}[Net]
Let \((X,\tau)\) be a topological space and \((I,\le)\) a directed set.  A \emph{net} in \(X\) is a map
\[
  x_\bullet\colon I \;\longrightarrow\; X,
  \quad i \mapsto x_i.
\]
\end{definition}

\begin{definition}[Convergence of a Net]
A net \((x_i)_S{i\in I}\) \emph{converges} to \(x\in X\), written \(x_i\to x\), if for every neighborhood \(U\ni x\) there exists \(i_0\in I\) such that
\[
  \forall\,i\ge i_0:\;x_i\in U.
\]
\end{definition}

\begin{definition}[Accumulation (Cluster) Point]
A point \(x\in X\) is an \emph{accumulation value} of the net \((x_i)_{i\in I}\) if for every neighborhood \(U\ni x\) and every \(i\in I\) there is \(j\ge i\) with
\[
  x_j\in U.
\]
Equivalently, \(x\) is the limit of some subnet of \((x_i)\).
\end{definition}

\begin{definition}[Final Map and Subnet]
Let \((I,\le_I)\) and \((J,\le_J)\) be directed sets.  A map \(h\colon I\to J\) is called \emph{final} if
\[
  \forall\,j\in J\;\;\exists\,i\in I:\;h(i)\ge_J j.
\]
Given a net \((x_j)_{j\in J}\), the composition \((x_{h(i)})_{i\in I}\) is called a \emph{subnet} of \((x_j)\).
\end{definition}

\begin{remark}
\begin{itemize}
  \item Every subnet of a convergent net converges to the same limit.
  \item Every accumulation value of a net is the limit of some subnet.
  \item Nets generalize sequences and allow characterization of continuity, compactness, and closure in arbitrary topological spaces.
\end{itemize}

\paragraph{Some easy characterizations of Sets} 
\begin{enumerate}[label = (\alph*)]
  \item $y \in \mathring{A} \iff y$ has a neighborhood contained in $A$. 
  \item $y \in \partial A \iff$ every neighborhood of $y$ contains a point of $A$ and 
  a point of $X \setminus A$. 
  \item $y \in \overline{A} \iff$ every neighborhood of $y$ contains a point in $A$. 
  \item $\overline{A} = A \cup \partial A = \mathring{A} \cup \partial A$ 
  \item $\mathring{A}, \operatorname{Ext}(A)$ are open, $\overline{A}, \partial A$ are closed. 
  \item \begin{align*}
    A \text{ open} &\iff A = \mathring{A} \\
    &\iff A \text{ contains a none of its boundary points}
  \end{align*}
  \item \begin{align*}
    A \text{ closed} &\iff A = \overline{A} \\
    &\iff A \text{ contains all its boundary points}
  \end{align*}
\end{enumerate}

\subsection{Subspaces, Quotient Spaces} 

% ===== Subspace & Embeddings =====
\begin{definition}[Subspace (Induced) Topology]
Let \((X,\tau)\) be a topological space and \(A\subseteq X\).  The \emph{subspace topology} (or \emph{induced} or \emph{relative topology}) on \(A\) is
\[
  \tau|_A \;=\;\{\,U\subseteq A : \exists\,V\in\tau,\;U = A\cap V\}.
\]
The space \((A,\tau|_A)\) is called a \emph{subspace} of \(X\).
\end{definition}

\begin{definition}[Embedding]
A continuous map \(f\colon (Y,\sigma)\to (X,\tau)\) is an \emph{embedding} if \(f\) is injective and the induced map 
\[
  f\colon Y \;\xrightarrow{\cong}\; f(Y)
\]
is a homeomorphism onto its image \(f(Y)\) equipped with the subspace topology \(\tau|_{f(Y)}\).
\end{definition}

\begin{proposition}[Universal Property of an Embedding]
Let \(i\colon Y\to X\) be an injective set map between topological spaces.  The following are equivalent:
\begin{enumerate}
  \item \(i\) is an embedding.
  \item For every topological space \(Z\) and every set map \(g\colon Z\to Y\), \(g\) is continuous if and only if the composition \(i\circ g\colon Z\to X\) is continuous.
\end{enumerate}
\end{proposition}

\begin{remark}
\begin{itemize}
  \item The inclusion map \(i\colon A\hookrightarrow X\) of a subspace \(A\subseteq X\) is always continuous.
  \item If \(A\subset B\subset X\) with \(A\) closed in \(B\) and \(B\) closed in \(X\), then \(A\) is closed in \(X\).  Similarly for open subspaces.
  \item An open (or closed) subset of a subspace \(B\subseteq X\) need not be open (or closed) in \(X\).
\end{itemize}
\end{remark}

% ===== Retracts & Sections =====
\begin{definition}[Retract]
Let \((X,\tau)\) be a topological space and \(A\subseteq X\).  A continuous map
\[
  r\colon X\;\longrightarrow\;A
\]
is called a \emph{retraction} if \(r|_{A} = \mathrm{id}_A\).  In this case \(A\) is called a \emph{retract} of \(X\).
\end{definition}

\begin{definition}[Section]
Let \(p\colon E\to B\) be a continuous surjection.  A continuous map
\[
  s\colon B\;\longrightarrow\;E
\]
is called a \emph{section} of \(p\) if \(p\circ s = \mathrm{id}_B\).  Equivalently, \(s\) is an embedding of \(B\) into \(E\).
\end{definition}

% ===== Quotient Topology =====
\begin{definition}[Quotient Topology]
Let \(f\colon X\to Y\) be a surjective map of spaces.  The \emph{quotient topology} on \(Y\) is
\[
  \tau_Y \;=\;\{\,U\subseteq Y : f^{-1}(U)\in\tau_X\}.
\]
Then \(f\) is continuous by construction.  If moreover
\[
  U\subseteq Y\quad\Longleftrightarrow\quad f^{-1}(U)\text{ is open in }X,
\]
we call \(f\) a \emph{quotient map} or \emph{identification map}, and \(Y\) a \emph{quotient space} of \(X\).
\end{definition}

\begin{definition}[Saturated Subset]
Given a surjection \(f\colon X\to Y\), a subset \(A\subseteq X\) is called \emph{saturated} (with respect to \(f\)) if it is a union of fibres of \(f\), i.e.\ \(A = f^{-1}(f(A))\).
\end{definition}

\begin{definition}[Quotient by an Equivalence Relation]
If \(\sim\) is an equivalence relation on \(X\), the set of equivalence classes \(X/\!\sim\) is given the quotient topology via the canonical projection
\[
  p\colon X\;\longrightarrow\;X/\!\sim,\quad p(x)=[x].
\]
\end{definition}

\begin{definition}[Collapsing a Subset]
For \(A\subseteq X\), the space \(X/A\) denotes the quotient obtained by identifying all of \(A\) to a single point (and leaving other points distinct), equipped with the quotient topology of the projection \(X\to X/A\).
\end{definition}

\begin{proposition}[Universal Property of Quotient Maps]
Let \(f\colon X\to Y\) be a surjective map of topological spaces.  Then \(f\) is a quotient map if and only if for every topological space \(Z\) and every map \(g\colon Y\to Z\),
\[
  g\text{ is continuous}
  \quad\Longleftrightarrow\quad
  g\circ f\colon X\to Z\text{ is continuous.}
\]
\end{proposition}

% ===== Adjunction (Pushout) Spaces =====
\begin{definition}[Adjunction Space / Pushout]
Let \(j\colon A\hookrightarrow X\) be the inclusion of a subspace and \(f\colon A\to Y\) a continuous map.  The \emph{adjunction space} or \emph{pushout} 
\[
  Z \;=\; Y\;\cup_{\,f}\;X
\]
is the quotient of the disjoint union \(X\sqcup Y\) by the relations
\[
  a\sim f(a),
  \quad
  \forall\,a\in A.
\]
It fits into the commutative square
\[
  \begin{tikzcd}
    A \arrow[r, "f"] \arrow[d, hook, "j"'] & Y \arrow[d, hook, "J"] \\
    X \arrow[r, "F"'] & Z
  \end{tikzcd}
\]
which is a pushout in \(\mathbf{Top}\).
\end{definition}

\begin{proposition}
Let \(j\colon A\hookrightarrow X\) be a \emph{closed embedding} and form the adjunction \(Z=Y\cup_f X\).  Then:
\begin{enumerate}
  \item \(J\colon Y\to Z\) is a closed embedding.
  \item \(F\colon X\setminus A \to Z\) is an open embedding.
  \item If \(X,Y\) are \(T_1\) (resp.\ \(T_4\)), then \(Z\) is \(T_1\) (resp.\ \(T_4\)).
  \item If \(f\) is a quotient map, then \(F\) is a quotient map.
\end{enumerate}
\end{proposition}

\begin{proposition}
The adjunction \(Z=Y\cup_f X\) is Hausdorff provided that:
\[
  Y \text{ is Hausdorff}, 
  \quad
  X \text{ is regular}, 
  \quad
  A \text{ is a retract of an open neighbourhood in }X.
\]
\end{proposition}

\end{document}