%===============================================
%   rhnotes.tex — Framework for RH-coding Notes
%===============================================
\documentclass[11pt,a4paper]{article}

%----- ENHANCED TYPOGRAPHY -----
\usepackage[utf8]{inputenc}
\usepackage[T1]{fontenc}
\usepackage{lmodern}        % clean vector font
\usepackage{microtype}      % better justification & kerning
\usepackage{palatino} 
\usepackage{braket}    % Palatino for text & math

%----- PAGE LAYOUT -----
\usepackage{geometry}
\geometry{top=1in, bottom=1in, left=1in, right=1in}
\usepackage{setspace}
\onehalfspacing  % 1.5 line spacing

%----- FANCY HEADERS & FOOTERS -----
\usepackage{fancyhdr}
\pagestyle{fancy}
\fancyhf{}
% page number outside, header text inside
\fancyhead[LE]{\small Georges Khater}
\fancyhead[RE]{\small Notes on RH-coding}
\fancyhead[LO]{\small \rightmark}
\fancyhead[RO]{\small \leftmark}
\renewcommand{\headrulewidth}{0.4pt}
\renewcommand{\footrulewidth}{0pt}

% make sections feed into \leftmark/\rightmark
\renewcommand{\sectionmark}[1]{\markboth{#1}{}}
\renewcommand{\subsectionmark}[1]{\markright{#1}}

%----- SECTION NUMBERING & TOC DEPTH -----
\setcounter{secnumdepth}{3}  % number down to \subsubsection
\setcounter{tocdepth}{2}     % show ToC down to \subsection

%----- AMS MATH & THEOREM STYLES -----
\usepackage{amsmath,amssymb,mathtools}
\usepackage{amsthm}

% definitions, examples, remarks upright
\theoremstyle{definition}
\newtheorem{definition}{Definition}[section]
\newtheorem{example}[definition]{Example}
\newtheorem{remark}[definition]{Remark}

% theorems, lemmas, corollaries italic
\theoremstyle{plain}
\newtheorem{theorem}[definition]{Theorem}
\newtheorem{lemma}[definition]{Lemma}
\newtheorem{proposition}[definition]{Proposition}
\newtheorem{corollary}[definition]{Corollary}

% unnumbered proof environment
\theoremstyle{remark}

%----- OTHER PACKAGES -----
\usepackage{graphicx}
\usepackage{tikz}
\usetikzlibrary{calc, matrix, decorations.pathreplacing, positioning}
\usepackage{hyperref}
\hypersetup{colorlinks,
linkcolor=blue, citecolor=purple, urlcolor=teal}
\usepackage{enumitem}
\setlist[itemize]{nosep, left=1.5em}
\usepackage{booktabs}
\usepackage{listings}
\lstset{
basicstyle=\ttfamily\small,
numbers=left,
numbersep=5pt,
frame=single,
breaklines=true
}
\usepackage{xcolor}
\definecolor{shade}{HTML}{F5F5F5}
\usepackage{float}
%----- CUSTOM MACROS -----
\newcommand{\F}{\mathbb{F}}
\newcommand{\code}[1]{\texttt{#1}}
\newcommand{\bsc}{\mathrm{BSC}}
\newcommand{\dist}[2]{d\bigl(#1,#2\bigr)}
\newcommand{\R}{\mathbb{R}}

%----- TITLE METADATA -----
\title{\LARGE\bfseries Algebraic Topology}
\author{Georges Khater \\ \small American University of Beirut}
\date{\today}

%===============================================
\begin{document}
\maketitle
\tableofcontents
\bigskip

\section{Introduction} 
In the first section \ref{sec:Topological-spaces} of this document, I will be summarizing all the useful notions of Point set topology that might be helpful for Algebraic topology. 
These notes will be taken mostly from the first chapter of \emph{Dieck's, Algebraic Topology} but also in part from  \emph{Martin Furter}'s playlist. Note that these are in no way a complete 
course / review on Point set topology, but should be useful for anyone taking a point set topology course wanting to study Algebraic Topology in parallel. I highly recommend doing all of \emph{Dieck's} 
exercises, as they tend to be useful and not too time consuming. 

\section{Topological Spaces} \label{sec:Topological-spaces}

\subsection{Basic Notions} 
%--------- Topology ---------
\begin{definition}[Topology]
Let \(X\) be a set.  A {\it topology} \(\tau\) on \(X\) is a collection \(\tau\subseteq\mathcal P(X)\) such that:
\begin{enumerate}
  \item \(\emptyset\in\tau\) and \(X\in\tau\).
  \item Any union of elements of \(\tau\) lies in \(\tau\).
  \item Any finite intersection of elements of \(\tau\) lies in \(\tau\).
\end{enumerate}
\end{definition}

\noindent\textbf{Example (\(\R\) with the usual metric).}
Let \(d(x,y)=|x-y|\).  Then
\[
  \tau_d \;=\;\{\,U\subseteq\R : (\forall x\in U)\;\exists\varepsilon>0:\;(x-\varepsilon,x+\varepsilon)\subseteq U\}
\]
is a topology on \(\R\).

%--------- Topological Space ---------
\begin{definition}[Topological Space]
A {\it topological space} is a pair \((X,\tau)\) where \(X\) is a set and \(\tau\) is a topology on \(X\).  Elements of \(\tau\) are called {\it open sets}.
\end{definition}

\noindent\textbf{Example.}
\((\R,\tau_d)\) is a topological space.

%--------- Basis ---------
\begin{definition}[Basis]
Let \((X,\mathcal O)\) be a topological space.  A subcollection \(\mathcal B \subseteq \mathcal O\) is called a \emph{basis} for the topology \(\mathcal O\) if for every \(U \in \mathcal O\) there exists a subfamily \(\mathcal B_U \subseteq \mathcal B\) such that
\[
  U \;=\; \bigcup_{B \in \mathcal B_U} B.
\]
\end{definition}

\noindent\textbf{Example (Basis for \(\tau_d\)).}
\[
  \mathcal B \;=\;\{(a,b)\subset\R : a<b\}
\]
is a basis: every open set in \(\tau_d\) is a union of open intervals.

%--------- Subbasis ---------
\begin{definition}[Subbasis]
Let \((X,\mathcal O)\) be a topological space.  A collection \(\mathcal S \subseteq \mathcal O\) is called a \emph{subbasis} for the topology \(\mathcal O\) if the family of all finite intersections of elements of \(\mathcal S\),
\[
  \bigl\{\,S_{1}\cap\cdots\cap S_{n} : n\ge1,\;S_{i}\in\mathcal S\bigr\},
\]
is a basis for \(\mathcal O\).
\end{definition}

\noindent\textbf{Example (Subbasis for \(\tau_d\)).}
\[
  \mathcal S
  = \{\,(-\infty,b)\;:\;b\in\R\}\;\cup\;\{\, (a,\infty)\;:\;a\in\R\}.
\]
Finite intersections give \((a,b)\), and unions of those recover every open set.

\begin{definition}[Continous map] 
  A map $f \cdot X \to Y$ between topological spaces is \emph{continuous} if the pre-image $f^{-1}(V)$ of each open set $V$ of $Y$ is open in $X$. 
  Dually: A map is continuous if the pre-image of each closed set is closed.   
\end{definition}

\noindent\textbf{Example $id(X)$} The identity $id(X) \colon X \to X$ is always continous. 

%--------- Initial (Weak) Topology ---------
\begin{definition}[Initial (Weak) Topology]
Let \(X\) be a set and \(\{\,f_i:X\to Y_i\}_{i\in I}\) a family of maps into topological spaces \((Y_i,\tau_i)\).  
The {\it initial topology} on \(X\) is the coarsest topology making every \(f_i\) continuous.  Equivalently, it is generated by the subbasis
\[
  \{\,f_i^{-1}(U) : i\in I,\;U\in\tau_i\}.
\]
\end{definition}

\begin{definition}[Homeomorphism]
  A \emph{homeomorphism} $f \colon X \to Y$ is a continuous map with a continuous inverse 
  $g \colon Y \to X$. 

  Spaces $X$ and $Y$ are \emph{homeomorphic} if there exists a homeomorphism between them. 
\end{definition}

\begin{definition}[Open and closed maps]
  A map $f \colon X \to Y$ between spaces is \emph{open (closed)} if the image of each open (closed) set is again open (closed).
\end{definition}

We fix a topological space $X$ and a subset $A$. 
\begin{definition}[Closure and Density]
  The intersection of the closed sets which contains $A$ is denoted $\overline{A}$ 
  and called \emph{closure} of $A$ in $X$. 

  A set $A$ is \emph{dense} in $X$ if $\overline{A} = X$. 
\end{definition}

\begin{definition}[Interior and Boundary]
  The \emph{interior} of $A$ is the union of the open sets contained in $A$. 

  The \emph{boundary} of $A$ in $X$ is 
  $$\operatorname{Bd}(A) = \overline{A} \cap \left(\overline{X \setminus A}\right)$$
\end{definition}

% ===== Neighbourhoods =====
\begin{definition}[Open Neighbourhood]
Let \((X,\tau)\) be a topological space and \(A\subseteq X\).  An \emph{open neighbourhood} of \(A\) is any open set \(U\in\tau\) with \(A\subseteq U\).
\end{definition}

\begin{definition}[Neighbourhood]
A set \(B\subseteq X\) is a \emph{neighbourhood} of \(A\subseteq X\) if there exists an open neighbourhood \(U\in\tau\) of \(A\) such that \(U\subseteq B\).
\end{definition}

\begin{definition}[Neighbourhood Basis]
A collection \(\mathcal N_x\) of neighbourhoods of a point \(x\in X\) is a \emph{neighbourhood basis} at \(x\) if for every neighbourhood \(B\) of \(x\) there is some \(N\in\mathcal N_x\) with \(N\subseteq B\).
\end{definition}

% ===== Continuity via Neighbourhoods =====
\begin{definition}[Continuity at a Point]
A map \(f\colon X\to Y\) between topological spaces is \emph{continuous at \(x\in X\)} if for every neighbourhood \(V\) of \(f(x)\) in \(Y\) there exists a neighbourhood \(U\) of \(x\) in \(X\) such that
\[
  f(U)\;\subseteq\;V.
\]
\end{definition}

% ===== Comparing Topologies =====
\begin{definition}[Finer and Coarser Topologies]
Let \(\mathcal O_1,\mathcal O_2\) be two topologies on the same set \(X\).  We say \(\mathcal O_2\) is \emph{finer} than \(\mathcal O_1\) (and \(\mathcal O_1\) is \emph{coarser} than \(\mathcal O_2\)) if and only if the identity map
\[
  \mathrm{id}\colon (X,\mathcal O_2)\;\longrightarrow\;(X,\mathcal O_1)
\]
is continuous.
\end{definition}

% ===== Intersection of Topologies =====
\begin{proposition}[Intersection of Topologies]
If \(\{\mathcal O_j\}_{j\in J}\) is any family of topologies on \(X\), then
\[
  \bigcap_{j\in J}\mathcal O_j
\]
is itself a topology on \(X\).
\end{proposition}

% ===== Separation Axioms =====
\begin{definition}[T\textsubscript{1} Space]
A topological space \((X,\tau)\) is called a \emph{T\textsubscript{1} space} if every singleton \(\{x\}\subseteq X\) is closed.
\end{definition}

\begin{definition}[T\textsubscript{2} / Hausdorff Space]
A space \((X,\tau)\) is \emph{T\textsubscript{2}} (or \emph{Hausdorff}) if for every pair of distinct points \(x,y\in X\) there exist disjoint open neighbourhoods \(U\ni x\) and \(V\ni y\) with \(U\cap V=\varnothing\).
\end{definition}

\begin{definition}[T\textsubscript{3} / Regular Space]
\((X,\tau)\) is \emph{regular} if it is T\textsubscript{1} and whenever \(x\in X\) and \(A\subseteq X\) is closed with \(x\notin A\), there exist disjoint open sets \(U\ni x\) and \(V\supseteq A\) such that \(U\cap V=\varnothing\).
\end{definition}

\begin{definition}[T\textsubscript{4} / Normal Space]
\((X,\tau)\) is \emph{normal} if it is T\textsubscript{1} and whenever \(A,B\subseteq X\) are disjoint closed sets, there exist disjoint open sets \(U\supseteq A\) and \(V\supseteq B\) with \(U\cap V=\varnothing\).
\end{definition}

\begin{definition}[Completely Regular Space]
\((X,\tau)\) is \emph{completely regular} if it is Hausdorff (T\textsubscript{2}) and for every closed set \(A\subseteq X\) and point \(x\notin A\), there exists a continuous map \(f\colon X\to [0,1]\) with \(f(x)=1\) and \(f(A)=\{0\}\).
\end{definition}

% ===== Key Theorems =====
\begin{theorem}[Urysohn Lemma]
Let \(X\) be a normal (T\textsubscript{4}) space, and let \(A,B\subseteq X\) be disjoint closed sets.  Then there exists a continuous function
\[
  f\colon X\;\longrightarrow\;[0,1]
  \quad\text{such that}\quad
  f(A)=\{0\},\quad f(B)=\{1\}.
\]
\end{theorem}

\begin{theorem}[Tietze Extension Theorem]
Let \(X\) be a normal (T\textsubscript{4}) space and \(A\subseteq X\) a closed subset.  Then any continuous map
\[
  f\colon A \;\longrightarrow\;[0,1]
\]
admits a continuous extension
\(\tilde f\colon X\to[0,1]\) with \(\tilde f|_A = f\).
\end{theorem}

% ===== First-Countable =====
\begin{definition}[First-Countable]
A topological space \((X,\tau)\) is said to be \emph{first-countable} if for each point \(x\in X\) there exists a countable collection of neighbourhoods 
\[
  \{\,U_{n}(x)\}_{n=1}^\infty
\]
such that for every neighbourhood \(V\) of \(x\), some \(U_n(x)\subseteq V\).  Equivalently, each point has a countable local basis.
\end{definition}

\begin{definition}
  A toplogical space $(X, \tau)$ is said to be \emph{second-countable} if there exists a countable 
  basis for its topology
\end{definition}

% ===== Key Points: Metrics ↔ Topology =====
\begin{itemize}
  \item \textbf{Metric induces a topology:}  
    In a metric space \((X,d)\), the open balls 
    \(\;U_\varepsilon(x)=\{\,y:d(x,y)<\varepsilon\}\)
    form a basis for the topology \(\tau_d\).

  \item \textbf{Metrizable spaces:}  
    A space \((X,\tau)\) is \emph{metrizable} if \(\tau=\tau_d\) for some metric \(d\).  
    Every metrizable space is first-countable (e.g.\ use the countable balls \(U_{1/n}(x)\)).

  \item \textbf{Properties of Metric spaces:} Metric spaces respect all sepration properties $T_1, \cdots, T_4$. 
\end{itemize}

\begin{definition}[Directed Set]
A \emph{directed set} is a pair \((I,\le)\) where \(\le\) is a reflexive, transitive relation on \(I\) such that
\[
  \forall\,i,j\in I\;\;\exists\,k\in I:\;i\le k\ \text{and}\ j\le k.
\]
\end{definition}

\begin{definition}[Net]
Let \((X,\tau)\) be a topological space and \((I,\le)\) a directed set.  A \emph{net} in \(X\) is a map
\[
  x_*\colon I \;\longrightarrow\; X,
  \quad i \mapsto x_i.
\]
\end{definition}

\begin{definition}[Convergence of a Net]
A net \((x_i)_S{i\in I}\) \emph{converges} to \(x\in X\), written \(x_i\to x\), if for every neighborhood \(U\ni x\) there exists \(i_0\in I\) such that
\[
  \forall\,i\ge i_0:\;x_i\in U.
\]
\end{definition}

\begin{definition}[Accumulation (Cluster) Point]
A point \(x\in X\) is an \emph{accumulation value} of the net \((x_i)_{i\in I}\) if for every neighborhood \(U\ni x\) and every \(i\in I\) there is \(j\ge i\) with
\[
  x_j\in U.
\]
Equivalently, \(x\) is the limit of some subnet of \((x_i)\).
\end{definition}

\begin{definition}[Final Map and Subnet]
Let \((I,\le_I)\) and \((J,\le_J)\) be directed sets.  A map \(h\colon I\to J\) is called \emph{final} if
\[
  \forall\,j\in J\;\;\exists\,i\in I:\;h(i)\ge_J j.
\]
Given a net \((x_j)_{j\in J}\), the composition \((x_{h(i)})_{i\in I}\) is called a \emph{subnet} of \((x_j)\).
\end{definition}

\begin{remark}
\begin{itemize}
  \item Every subnet of a convergent net converges to the same limit.
  \item Every accumulation value of a net is the limit of some subnet.
  \item Nets generalize sequences and allow characterization of continuity, compactness, and closure in arbitrary topological spaces.
\end{itemize}

\paragraph{Some easy characterizations of Sets} 
\begin{enumerate}[label = (\alph*)]
  \item $y \in \mathring{A} \iff y$ has a neighborhood contained in $A$. 
  \item $y \in \partial A \iff$ every neighborhood of $y$ contains a point of $A$ and 
  a point of $X \setminus A$. 
  \item $y \in \overline{A} \iff$ every neighborhood of $y$ contains a point in $A$. 
  \item $\overline{A} = A \cup \partial A = \mathring{A} \cup \partial A$ 
  \item $\mathring{A}, \operatorname{Ext}(A)$ are open, $\overline{A}, \partial A$ are closed. 
  \item \begin{align*}
    A \text{ open} &\iff A = \mathring{A} \\
    &\iff A \text{ contains a none of its boundary points}
  \end{align*}
  \item \begin{align*}
    A \text{ closed} &\iff A = \overline{A} \\
    &\iff A \text{ contains all its boundary points}
  \end{align*}
\end{enumerate}

\subsection{Subspaces, Quotient Spaces} 

% ===== Subspace & Embeddings =====
\begin{definition}[Subspace (Induced) Topology]
Let \((X,\tau)\) be a topological space and \(A\subseteq X\).  The \emph{subspace topology} (or \emph{induced} or \emph{relative topology}) on \(A\) is
\[
  \tau|_A \;=\;\{\,U\subseteq A : \exists\,V\in\tau,\;U = A\cap V\}.
\]
The space \((A,\tau|_A)\) is called a \emph{subspace} of \(X\).
\end{definition}

\begin{definition}[Embedding]
A continuous map \(f\colon (Y,\sigma)\to (X,\tau)\) is an \emph{embedding} if \(f\) is injective and the induced map 
\[
  f\colon Y \;\xrightarrow{\cong}\; f(Y)
\]
is a homeomorphism onto its image \(f(Y)\) equipped with the subspace topology \(\tau|_{f(Y)}\).
\end{definition}

\begin{proposition}[Universal Property of an Embedding]
Let \(i\colon Y\to X\) be an injective set map between topological spaces.  The following are equivalent:
\begin{enumerate}
  \item \(i\) is an embedding.
  \item For every topological space \(Z\) and every set map \(g\colon Z\to Y\), \(g\) is continuous if and only if the composition \(i\circ g\colon Z\to X\) is continuous.
\end{enumerate}
\end{proposition}

\begin{remark}
\begin{itemize}
  \item The inclusion map \(i\colon A\hookrightarrow X\) of a subspace \(A\subseteq X\) is always continuous.
  \item If \(A\subset B\subset X\) with \(A\) closed in \(B\) and \(B\) closed in \(X\), then \(A\) is closed in \(X\).  Similarly for open subspaces.
  \item An open (or closed) subset of a subspace \(B\subseteq X\) need not be open (or closed) in \(X\).
\end{itemize}
\end{remark}

% ===== Retracts & Sections =====
\begin{definition}[Retract]
Let \((X,\tau)\) be a topological space and \(A\subseteq X\).  A continuous map
\[
  r\colon X\;\longrightarrow\;A
\]
is called a \emph{retraction} if \(r|_{A} = \mathrm{id}_A\).  In this case \(A\) is called a \emph{retract} of \(X\).
\end{definition}

\begin{definition}[Section]
Let \(p\colon E\to B\) be a continuous surjection.  A continuous map
\[
  s\colon B\;\longrightarrow\;E
\]
is called a \emph{section} of \(p\) if \(p\circ s = \mathrm{id}_B\).  Equivalently, \(s\) is an embedding of \(B\) into \(E\).
\end{definition}

% ===== Quotient Topology =====
\begin{definition}[Quotient Topology]
Let \(f\colon X\to Y\) be a surjective map of spaces.  The \emph{quotient topology} on \(Y\) is
\[
  \tau_Y \;=\;\{\,U\subseteq Y : f^{-1}(U)\in\tau_X\}.
\]
Then \(f\) is continuous by construction.  If moreover
\[
  U\subseteq Y\quad\Longleftrightarrow\quad f^{-1}(U)\text{ is open in }X,
\]
we call \(f\) a \emph{quotient map} or \emph{identification map}, and \(Y\) a \emph{quotient space} of \(X\).
\end{definition}

\begin{definition}[Saturated Subset]
Given a surjection \(f\colon X\to Y\), a subset \(A\subseteq X\) is called \emph{saturated} (with respect to \(f\)) if it is a union of fibres of \(f\), i.e.\ \(A = f^{-1}(f(A))\).
\end{definition}

\begin{definition}[Quotient by an Equivalence Relation]
If \(\sim\) is an equivalence relation on \(X\), the set of equivalence classes \(X/\!\sim\) is given the quotient topology via the canonical projection
\[
  p\colon X\;\longrightarrow\;X/\!\sim,\quad p(x)=[x].
\]
\end{definition}

\begin{definition}[Collapsing a Subset]
For \(A\subseteq X\), the space \(X/A\) denotes the quotient obtained by identifying all of \(A\) to a single point (and leaving other points distinct), equipped with the quotient topology of the projection \(X\to X/A\).
\end{definition}

\begin{proposition}[Universal Property of Quotient Maps]
Let \(f\colon X\to Y\) be a surjective map of topological spaces.  Then \(f\) is a quotient map if and only if for every topological space \(Z\) and every map \(g\colon Y\to Z\),
\[
  g\text{ is continuous}
  \quad\Longleftrightarrow\quad
  g\circ f\colon X\to Z\text{ is continuous.}
\]
\end{proposition}

% ===== Adjunction (Pushout) Spaces =====
\begin{definition}[Adjunction Space / Pushout]
Let \(j\colon A\hookrightarrow X\) be the inclusion of a subspace and \(f\colon A\to Y\) a continuous map.  The \emph{adjunction space} or \emph{pushout} 
\[
  Z \;=\; Y\;\cup_{\,f}\;X
\]
is the quotient of the disjoint union \(X\sqcup Y\) by the relations
\[
  a\sim f(a),
  \quad
  \forall\,a\in A.
\]
and 
$$x \sim x \quad \forall \, x \in X \setminus A, \quad y \sim y \quad \forall\, y \in Y \setminus f(A)$$
This also gives us a canonical $F \colon X \to Z$ define by 
$$F(x) = [x]$$
\end{definition}

\begin{proposition}
Let \(j\colon A\hookrightarrow X\) be a \emph{closed embedding} and form the adjunction \(Z=Y\cup_f X\).  Then:
\begin{enumerate}
  \item \(J\colon Y\to Z\) is a closed embedding.
  \item \(F\colon X\setminus A \to Z\) is an open embedding.
  \item If \(X,Y\) are \(T_1\) (resp.\ \(T_4\)), then \(Z\) is \(T_1\) (resp.\ \(T_4\)).
  \item If \(f\) is a quotient map, then \(F\) is a quotient map.
\end{enumerate}
\end{proposition}

\begin{proposition}
The adjunction \(Z=Y\cup_f X\) is Hausdorff provided that:
\[
  Y \text{ is Hausdorff}, 
  \quad
  X \text{ is regular}, 
  \quad
  A \text{ is a retract of an open neighbourhood in }X.
\]
\end{proposition}

\subsection{Product and Sums} 

Let $((X_j, \mathcal{O}_j))$ be a family of topological spaces. We define the product set 
$$X = \prod_{j \in J} X_j = (x_j \mid j \in J)$$ 
We also define the projection map 
$$\operatorname{pr}_i \colon \: X \to X_i, \ (x_j) \to x_i$$ 
Let $X_j, Y_j$ be topological spaces and 
$$f_j \colon \: X_j \to Y_j$$
The product map $\prod f_j \colon \: \prod X_j \to \prod Y_j$ is defined 
$$(x_j) \to (f_j(x_j))$$

\begin{definition}
  We define the \emph{product topology} $\mathcal{O}$ on $X$ to be the toplogy generated by the family of all preimages 
  $$\operatorname{pr}^{-1} (U_j) \quad U_j \text{ open in } X_j$$
  Note that this is the coarsest topology for which all projections are continuous. 
  We call $(X, \mathcal{O})$ the \emph{toplogical product} of the spaces $(X_j, \mathcal{O}_j)$
\end{definition}

\begin{proposition}
  A set map $f \colon \: X \to Y$ is continuous if and only if all maps 
  $\operatorname{pr}_j \circ f$ are continous. The product $f = \prod_j f_j$ of continous maps is continuous. 
\end{proposition}

\begin{proposition}
  Let $f \colon \: X \to Y$ be surjective, continuous and open. Then $Y$ is separated (Hausdorff) iff 
  $$R = \{(x_1, x_2) \mid f(x_1) = f(x_2)\}$$
  is closed in $X \times X$. 
\end{proposition}

Let $(X_j)$ be a family of non-empty pairwise disjoint spaces. The set 
$$\mathcal{O} = \{U \subset \sqcup X_j \mid U \cap X_j \subset X_j \text{ open for all } j\}$$
is a toplogy on the disjoint union $\sqcup X_j$. A sum of two spaces is denoted $X_1 + X_2$. 

\begin{proposition}
  \begin{itemize}
    \item The subspace topology of $X_j$ in $\sqcup X_j$ is the original toplogy. 
    \item Let $X$ be the union of the familiy $(X_j)$ of pairwise disjoin subsets, Then $X$ is the 
    toplogical sum of the subspaces $X_j$ iff the $X_j$s are open. 
    \item $f \colon \sqcup X_j \to Y$ is continous if and only if each $f_{\mid X_j} \colon X_j \to Y$ is continous. 
  \end{itemize}
\end{proposition}

\begin{definition}[Clutching]
  An important method for the construction of spaces is to "paste" open subsets. 
  Let $(U_j \mid j \in J)$ be a family of sets. Assume that for each pair $(i, j) \in J \times J$ a subset $U^j_i \subset U_i$ is given, 
  as well as a bijection 
  $$g_i^j \colon \: U_i^j \to U_j^i$$
  We call the families $(U_j, U_j^k, g_j^k)$ a \emph{clutching datum} if: 
  \begin{enumerate}
    \item $U_j = U_j^j$ and $g_j^j = id$. 
    \item For each triple $(i,j,k) \in J \times J \times J$ the map $g_i^j$ induces a bijection 
    $$g_i^j \colon U_i^j \cap U_i^k \to U_j^i \cap U_j^k$$
    and $g_j^k \circ g_i^j = g_i^k$ holds, considered as maps from $U_i^j \cap U_i^k$ to $U_k^j \cap U_k^i$. \\
  \end{enumerate}  
  Note that this forces $g_i^j$ and $g_j^i$ to be inverses. 
\end{definition}

Given a clutching datum, we get an equivalence relation on the disjoint sum $\sqcup_{j \in J} U_j$: 
$$x \in U_i \sim y \in U_j \iff x \in U_i^j \land g_i^j(x) = y$$
Let $X$ be the set of equivalence classes, and let $h_i \colon U_i \to X$ be the map which sends 
$x \in U_i$ to its class. Clearly $h_i$ is injective, set $U(i) = Im(h_i)$, then 
$$U(i) \cap U(j) = h_i(U_i^j)$$

Conversely, assume that $X$ is a quotient of $\sqcup_{j \in J} U_j$ such that each $h_i \colon U_i \to X$ 
is injective with image $U(i)$. Let $U_i^j = h_i^{-1} \left(U(i) \cap U(j)\right)$ and 
and $g_i^j = h_j^{-1} \circ h_i \colon U_i^j \to U_j^i$. Then the $(U_i, U_i^j, g_i^j)$ are a clutching datum. 

\begin{proposition}
  Let $(U_i, U_i^j, g_i^j)$ be a clutching datum. Assume that the $U_i$ are topological spaces, 
  $U_i^j \subset U_i$ open subsets, and the $g_i^j \colon U_i^j \to U_j^i$ homeomorphisms. Let $X$ carry 
  the quotient toplogy with respect to the quotient map 
  $$p \colon \sqcup_{i \in J} U_j \to X$$
  Therefore: 
  \begin{itemize}
    \item The map $h_i$ is a homeomorphism onto an open subset of $X$ and $p$ is open. 
    \item Suppose the $U_i$ aer Hausdorff spaces. Then $X$ is a Hausdorff space iff for each pair $(i,j)$ the map 
    $$\gamma_i^j \colon U_I^j \to U_i \times U_j, x \to (x, g_i^j(x))$$
    is a closed embedding. 
  \end{itemize}

  This allows us to construct spaces by gluing "patches" of different spaces. 
\end{proposition}

\begin{definition}
  Suppose a space $X$ is the (not necessarily disjoint) union of subspaces $(X_j)$. We say $X$ carries the 
  \emph{colimit topology} with respect to this family if one of the equivalent statements hold: 
  \begin{itemize}
    \item The canonical map $\sqcup_{j \in J} X_j \to X$ (inclusion on each summand) is a quotient map. 
    \item $C$ is closed in $X$ iff $X_j \cap C$ is closed in $X_j$ for each $j$. 
    \item A set map $f \colon X \to Z$ into a space $Z$ is continuous if and only if the restrictions 
    $f_{\mid X_j} \colon X_j \to Z$ are continous. 
  \end{itemize}
\end{definition}

\paragraph{Example} Let $X$ be a set covered by a family $(X_j)$ of subsets. Suppose each $X_j$ 
carries a topology such that the subspace topologies of $X_i \cap X_j$ in $X_i$ and $X_j$ coincide 
and these subspaces are closed. Then there is a unique topology on $X$ which induces on $X_j$ the given topology. 
The space $X$ has the colimit topology with respect to the $X_j$.

\subsection{Topological Manifolds} 
\begin{definition}
  A space $M$ is \emph{locally Euclidian of dimension $n$} if every point $x \in M$ has a neighborhood 
  homeomorphic to an open ball $\mathbb{B}^n \subseteq \mathbb{R}^n$.  \\
  (Note that $\mathbb{B}^n \cong \mathbb{R}^n$)
\end{definition}

\begin{definition}
  An \emph{$n$-dimensional toplogical maniforld} is a second countable Hausdorff space that is locally Euclidian of 
  dimension $n$. 
\end{definition}

\begin{proposition}
  If $m \neq n$, a nonempty space cannot be both an $m$-manifold and an $n$-manifold. 
\end{proposition}

\paragraph{Charts and Atlases for Manifolds} 
\begin{definition}
  A \emph{coordinate domain} (or \emph{chart domain}) on an $n$-manifold $M$ is an open set
  $U \subseteq M$ together with a homeomorphism
  \[
    \varphi \colon U \;\xrightarrow{\;\cong\;}\; V,
  \]
  where $V$ is an open subset of $\mathbb{R}^n$.
\end{definition}

\begin{definition}
  A \emph{coordinate map} (or \emph{chart map}) on $M$ is the pair
  \[
    (U,\varphi),
  \]
  where $U\subseteq M$ is a coordinate domain and
  $\varphi \colon U \to V \subseteq \mathbb{R}^n$ is the corresponding homeomorphism.
\end{definition}

\begin{definition}
  An \emph{atlas} on $M$ is a collection of coordinate maps
  \[
    \{(U_i,\varphi_i)\}_{i\in I}
  \]
  such that
  \begin{enumerate}
    \item $\displaystyle \bigcup_{i\in I} U_i = M$, and
    \item for any overlap $U_i \cap U_j \neq \emptyset$, the transition map
      \[
        \varphi_j \circ \varphi_i^{-1} \;\colon\; \varphi_i(U_i \cap U_j)\;\longrightarrow\;\varphi_j(U_i \cap U_j)
      \]
      is a homeomorphism between open subsets of $\mathbb{R}^n$.
  \end{enumerate}
\end{definition}

\paragraph{Toplogical Manifolds with Boundary} 

\begin{definition}
  The \emph{Upper half-space} $\mathbb{H}^n \subseteq \mathbb{R}^n$ is 
  $$\mathbb{H} := \{(x_1, \cdots, x_n) \in \mathbb{R}^n \mid x_n \geq 0\}$$
\end{definition}

\begin{definition}
  An \emph{$n$-dimensional manifold with boundary} is a second countable Hausdorff space in which 
  every point has a neighborhood homeomorphic to an open subset of $\mathbb{R}^n$ or $\mathbb{H}^n$. 
\end{definition}

\begin{proposition}
  No point of a topological manifold with boundary is both a boundary point and an interior point of the manifold.
\end{proposition}

\subsection{Compact Spaces}
\begin{definition}
  A family $A = (A_j \mid j \in J)$ of subsets of $X$ is a \emph{covering} of $X$ if $X$ is the union of the 
  $A_j$. 

  A covering $B = (B_k)$ of $X$ is a \emph{refinement} of $A$ if for each $k \in K$ there exists $j \in J$ such that $B_k \subset A_j$.

  A covering is said to be open (closed) if each of the $A_j$s is open (closed). 
\end{definition}

\begin{definition}
  A covering $B = (B_k)_{k \in K}$ is a \emph{subcovering} of $A = (A_j)_{j \in J}$ if $K \subset J$ and $B_k = A_k$ for all $k \in K$. We say that 
  $B$ is \emph{finite} or \emph{countable} if $K$ is finite or countable. 

  A covering $A$ is \emph{locally finite} if each $x \in X$ has a neighborhood $U$ such that $U \cap A_j \neq \emptyset$ only for 
  a finite number of $j \in J$. It is called \emph{point-finite} if each $x \in X$ is contained only in a finite number of $A_j$. 
\end{definition}

\begin{definition}[Compact space]
  A space $X$ is \emph{compact} if every open covering has a finite subcovering. 
  Note that equivalently, every family of closed sets with empty intersection has a finite 
  subfamily of sets with empty intersection ($X$ is compact if and only if every collection of closed sets with the finite-intersection property has 
  a nonempty total intersection). 
\end{definition}

\begin{definition}
  A set $A$ in a space $X$ is \emph{relatively compact} if its closure is compact.
\end{definition}

\begin{proposition}
  \begin{itemize}
    \item A space $X$ is compact if and only each net in $X$ has a convergent subnet. 
    \item A discrete closed set in a compact space is finite. 
    \item Let $X$ be compact, $ A \subset X$ closed and $f \colon X \to Y$ continuous; therefore $A$ and $F(X)$ are 
    compact. 
  \end{itemize}
\end{proposition}

\begin{proposition}
  Let $B, C$ be compacts subsets of $X, Y$ respectively. Let $\mathcal{U}$ be a family of open subsets of 
  $X \times Y$ which cover $B \times C$. Then there exist open neighborhoods $U$ of $B$ and $V$ of $C$ such that $U \times V$ is covered by a finite 
  subfamily of $\mathcal{U}$. In particular, the product of two compact spaces is compact.

  Moreover, an arbitrary product of compact spaces is compact.
\end{proposition}

\begin{proposition}
  Let $B$ and $C$ be disjoint compact subsets of a Hausdorff space $X$. Then $B$ and $C$ has disjoint open neighborhoods. A compact Hausdorff space is normal. 
  A compact subset $C$ of a Hausdorff space $X$ is closed.   
\end{proposition}

\begin{proposition}
  A continuous map $f \colon X \to Y$ from a compact space into a Hausdorff space is closed.
  If, moreover, $f$ is injective (bijective), then $f$ is a homeomorphism. If $f$ is surjective, then $f$ is a quotient map. 
\end{proposition}

\begin{proposition}
  Let $X$ be a compact Hausdorff space and $f \colon X \to Y$ a quotient map. The following assertions are equivalent: 
  \begin{enumerate}
    \item $Y$ is Hausdorff
    \item $f$ is closed 
    \item $R = \{(x_1, x_2) \mid f(x_1) = f(x_2)\}$ is closed in $X \times X$. 
  \end{enumerate}
\end{proposition}

\begin{definition}
  A space if called \emph{locally compact} if each neighborhood of a point $x$ contains a 
  compact neighborhood. An open subset of a locally compact space is again locally compact. 
\end{definition}

Let $X$ be Hausdorff and assume that each point has a compact neighborhood. Let $U$ be a 
neighborhood of $x$ and $K$ a compact neighborhood. Since $K$ is normal, $K \cap U$ contains a closed neighborhood $L$ of $x$ 
in $K$. Therefore $X$ is locally compact, in particular a compact Hausdorff space is locally compact. 
If $X$ and $Y$ are locally compact, $X \times Y$ is locally compact. 

\begin{definition}
  Let $X$ be a topological space. An embedding $f \colon X \to Y$ is a \emph{compactification} of $X$ 
  if $Y$ is compact and $f(X)$ dense in $Y$. 

  In a general compactification $f \colon X \to Y$, one calls the points in $Y \setminus f(X)$ the points at infinity. 
\end{definition}

\begin{theorem}[one-point compactification]
  Let $X$ be a locally compat Hausdorff space. Up to homeomorphism, there exists 
  a unique compactification $f \colon X \to Y$ by a compact Hausdorff space such 
  that $Y \setminus f(X)$ consists of a single point. 
\end{theorem}

\begin{proposition}
  Let $K$ be a locally compact space such that $K$ is a union of compact subsets $(K_i)_{i \in \mathbb{N}}$. 
  Then there exists a sequence $(U_i)_{i \in \mathbb{N}}$ of open subsets with union $K$ such that 
  each $\overline{U}_i$ is compact and contained in $U_{i+1}$.  
\end{proposition}

\begin{theorem}
  Let the locally compact Hausdorff space $M \neq \emptyset$ be a union of closed subsets $M_n, n \in \mathbb{N}$. Then at least one 
  of the $M_n$ contains an interior point. 
\end{theorem}

\begin{definition}
  A subset $H$ of a space $G$ is called \emph{locally closed}, if each $x \in H$ has a neighborhood $V_x$ in $G$ 
  such that $H \cap V_x$ is closed in $G$. 
\end{definition}

\begin{proposition}
  \begin{enumerate}
    \item Let $A$ be locally closed in $X$. Then $A = U \cap C$ with $U$ open and $C$ closed. Conversely, if $X$ is regular, 
    then an intersection $U \cap C$, $U$ open $C$ closed, is locally closed. 

    \item A locally compact set in a Hausdorff space $X$ is locally closed. 
    \item A locally closed set $A$ in a locally compact space is locally compact. 
  \end{enumerate}
\end{proposition}

\subsection{Connectedness and Path-Connectedness}
\paragraph{Path-Connectedness}
\begin{definition}
  A \emph{path} in a topological space $X$ from $x$ to $y$ is a continuous map $u \colon [a,b] \to X$ 
  such that $u(a) = x$ and $u(b) = y$. We say that the path connects the point $u(a)$ and $u(b)$.
  
  We can reparametrize and use the unit interval as a source $[0,1] \to X$, $t \to u\left((1-t)a + t b\right)$.
  In general, we mostly use the unit interval. 
\end{definition}

\begin{definition}
  If $u \colon [0,1] \to X$ is a path from $x$ to $y$, then the \emph{inverse path} 
  $u^{-} \colon t \to w(1-t)$ is a path from $y$ to $X$. 
\end{definition}

\begin{definition}
  If $v \colon [0,1] \to X$ is another path from $y$ to $z$, then the \emph{product path} $u * v$, defined by 
  $$\begin{cases}
    u(2t) \quad &t \leq 1/2 \\
    v(2t - 1) \quad &t \geq 1/2
  \end{cases}$$
  is a pagh from $x$ to $x$. 
\end{definition}

\begin{definition}
  We also define the constant path $k_x$ with constant value $x$.
\end{definition}

Notice that being connectible by paths is an equivalence relation on $X$. An equivalence class is 
called a \emph{path component} of $X$. We denote by $\pi_0(X)$ the set of path components, and by $[x]$ the 
path component of the point $x$. Note that a continuous map $f \colon X \to Y$ induces a function 
$$\pi_0(f) \colon \pi_0(X) \to \pi_0(Y), \, [x] \to [f(x)]$$

\begin{definition}[Path-connected]
  A space is said to be \emph{path-connected} or \emph{0-connected} if it has one of the following equivalent properties: 
  \begin{enumerate}
    \item $\pi_0(X)$ consists of a single element. 
    \item Any two points can be joined by a path. 
    \item Any continuous map $f \colon \: \partial I = \{0,1\} \to X$ has a continuous extension 
    $F \colon I \to X$. \\
    (Where $\partial I$ represents the boundary of the set $I$)
  \end{enumerate}
\end{definition}

\paragraph{Connectedness}
\begin{definition}
  A space is \emph{connected} if it is not the topological sum of two non-empty subspaces.

  Thus $X$ is \emph{disconnected} if and only if $X$ Contains a subset $X$ which is open, closed and different 
  from $\emptyset$ and $X$. 
\end{definition}

\begin{definition}
  A \emph{Decomposition} of $X$ is a pair $U,V$ of open, non-empty, disjoint subsets with union $X$. 
\end{definition}

\begin{proposition}
  A space $X$ is disconnected if and only if there exists a continuous surjective map $f \colon X \to \{0,1\}$; a decomposition 
  is given by $U = f^{-1}(0), \, V = f^{-1}(1)$. 

  The continuous image of a connected space is connected. (In $\mathbb{R}$, 
  $A$ is connected iff $A$ is an interval). 
\end{proposition}

\begin{proposition}
  Let $(A_j)_{j \in J}$ be a family of connected subsets of $X$ such taht $A_i \cap A_j \neq \emptyset \quad \forall i, j \in J$. 
  Then $\bigcup_{j} A_j = Y$ is connected.

  Let $A$ be connected, and $A \subset B \subset \overline{A}$, then $B$ is connected. 
\end{proposition}

\begin{definition}[Component of $x$]
  The union of the connected sets in $X$ which contain $x$ is thus a closed connected subset. 
  We define this union to be the \emph{component of $x$} denoted $X(x)$. 

  Note that if $y \in X(x)$, then $X(y) = X(x)$. Moreover, the component of $X$ is a maximal connected subsets. Any space is 
  the disjoint union of its component. 
\end{definition}

\begin{definition}
  A space is said to be \emph{totally disconnected} if its components consists of single points.
\end{definition}

Since intervals are connected a path connected space is connected. Moreover a product $\prod_j X_j$ 
is connected if each $X_j$ is connected, the component of $(x_j)_j$ is the product of the 
components of the $x_j$. 

\begin{proposition}
  If $X$ is locally-path connected, then its connected components and its path-connected components coincide, 
  and each component is open. 
\end{proposition}

\section{Cell Complexes} 

\begin{definition}
  An \emph{n-cell} is a space that is homeomorphic to an open $n$-dimensional ball.
\end{definition}

An orientable surface $M_g$ of genus $g$ an be constructed from a polygon with $4g$ sides by 
identifying pairs of edges. The $4g$ edges of the polygon become a union of $2g$ circles in the surface, all 
intersecting in a single point. But notice that this surface can be constructed in stages: Start with a single point, 
Attach two $1$-cells to this point, then attach a $2$-cell. 

More specifically these complexes can be built by the following steps: 
\begin{itemize}
  \item $0$-skeleton $X^0$: Discrete point(s)
  \item $1$-skeleton $X^1$: add $D^1$ (Line(s))\\
  For each $D^1$ we have an attaching map 
  $$\varphi \colon \: S^0 = \partial D^1 \to X^0$$
  \item $2$-skeleton $X^2$: add $D^2$ (Disk(s)) \\
  According to an attaching map 
  $$\varphi \colon \; S^1 = \partial D^2 \to X^1$$
  \item $n$-skeleton $X^n$: add $D^n$ via maps 
  $$\varphi \colon \: S^{n-1} = \partial D^n \to X^{n-1}$$
\end{itemize}

We can generalize by constructing a space by the following procedure:
\begin{itemize}
  \item Start with a discrete set $X^0$, whose points are regarded as $0$-cells. 
  \item Given $X^{n-1}$, choose a family of attaching maps
  $$\varphi_\alpha \colon \: S^{n-1} \to X^{n-1} \quad \alpha \in A_n$$
  Form the quotient 
  $$X^n = \left(X^{n-1} \sqcup \bigsqcup_{\alpha \in A_n} D_\alpha^n\right) \big/ \left\{x \in \partial D_\alpha^n \sim \varphi_\alpha(x)\right\}$$

  \item One can either stop the process at a finite stage, setting $X = X^n$ for some $n < \infty$, or one can continue indefinitely, setting 
  $$X = \bigcup_n X^n$$
  In the latter case, $X$ is given the weak topology: $A \subset X$ is open iff $A \cap X^n$ is open (in $X^n$) for each $n$.
\end{itemize}

\begin{example}
  The sphere $S^n$ has the structure of a cell complex with just two cells $e^0$ and $e^n$, 
  then $n$ cell being attached by the constant map $S^{n-1} \to e^0$. This is equivalent to regarding $S^n$ as 
  the quotient space $D^n / \partial D^n$. 
\end{example}
\begin{definition}
  A space $X$ constructed by the above procedure is called a \emph{cell complex} or \emph{CW-complex} (Closure finite, Weak).
\end{definition}

\begin{definition}
  If $X = X^n$ for some $n$, then $X$ is said to be finite-dimensional, and the smallest such $n$ is the \emph{dimension} of $X$, the maximum 
  dimension of cells of $X$.
\end{definition}

\begin{definition}
  A \emph{subcomplex} of a cell complex $X$ is a closed subspace $A \subset X$ that is a union of cells of $X$.
  Since $A$ is closed, $A$ is a cell complex in its own right.
  
  We call a \emph{CW pair} a pair $(X,A)$ consisting of a cell complex $X$ and a subcomplex $A$. 
\end{definition}

\subsection{Real projective spaces} 

\begin{definition}
  The \emph{Real projective n-space $\mathbb{R}P^n$} is the space of all lines through the origin in $R^{n+1}$
\end{definition}

Notice that $\mathbb{R}P^n = \mathbb{RP^{n-1}} + D^n = D^0, D^1, D^2, \cdots D^n$. 

\section{Homology}
Homology assigns a sequence of abelian groups $H_n(X)$ to a space $X$ that measures its 
$n$ dimensional "holes". 
\begin{itemize}
  \item $H_0$: connected components 
  \item $H_1$: loops that don't bound a surface 
  \item $H_2$: surfaces that don't bound a volume 
  \item $\cdots$
\end{itemize}

\subsection{The idea of Homology} 
Suppose that we are working on a graph with edges $a,b,c,d$ going from vertex $x$ to vertex $y$. 
Let $C_1$ be the free abelian group with basis the edges $a,b,c,d$ and let $C_0$ be the free abelian group with 
basis the vertices $x,y$.
Then elements of $C_1$ are chains of edges, or $1$-dimensional chains, and elements of $C_0$ are linear combinations of 
vertices or $0$-dimensional chains. 

Define a homomorphism $\partial \colon C_1 \to C_0$ by sending each basis element $a,b,c,d$ to $y-x$ (the vertex at the end of the edge minus the one at the tail).
Thus we have that 
$$\partial (ka + lb + mc + nd) = (k + l + m + n)y - (k + l + m +n) x$$ 
So all cycles of $X_1$ are precisely the kernel of $\partial$. But notice that 
$$\{a-b, b-c, c-d\}$$
form a basis for this kernel. Therefore by means of these three basic cycles we convey 
geometric information that the graph $X_1$ has three visible 'holes'. 

Now attach a $2$-cell along the cycle $a - b$, producing a $2$-dimensional cell complex $X_2$. The boundary 
of the $2$-cell $A$ is therefore the cycle $a-b$, which can be retracted to a point. So $a-b$ no longer encloses a hole in $X_2$.
This suggests we form a quotient by factoring out the group generated by $a-b$. 
Algebraically, we can define now a pair of homomorphisms $C_2 \xrightarrow{\partial_2} C_1 \xrightarrow{\partial_1} C_0$ 
where $C_2$ is the infinite cyclic group generated by $A$ and $\partial_2(A) = a-b$. The map $\partial_1$ 
is the boundary homomorphism described abouve. We are therefore intersted in the quotient group $\ker \partial_1 / \operatorname{Im} \partial_2$.
Notice that in this quotient, cycles like $a-c$ and $b-c$ are equivalent, which is consistent with our homotopic intuition. 

Now suppose we enlarge $X_2$ to $X_3$ by attaching a second $2$-cell $B$ along $a-b$. $C_2$ becomes the group of linear combinations 
of $A$ and $B$, and $\partial_2 \colon C_2 \to C_1$ sends both $A$ and $B$ to $a-b$. 
$H_1(X_3)$ is the same as for $X_2$, but $\partial_2$ has a nontrivial kernel $\langle A-B \rangle$. 
We view $A-B$ as a $2$-dimensional cycle, generating the homology group $H_2(X_3) = \ker \partial_2 \cong \mathbb{Z}$. 
This spherical cycle detects the presence of a 'hole' in $X_3$, however this hole is enclosed by a sphere rather than a circle.

Finally, let's construct $X_4$ by attaching a $3$-cell $C$ along the $2$-sphere formed by $A$ and $B$. 
This creates a chain group $C_3$ generated by $C$, we can define a boundary homomorphism $\partial_3 \colon C_3 \to C_2$ with $\partial_3(C) = A - B$.
Now we have a sequence of homomorphisms $C_3 \xrightarrow{\partial_3} C_2 \xrightarrow{\partial_2} C_1 \xrightarrow{\partial_1} C_0$ 
and the quotient $H_2(X_4) = \ker \partial_2 / \operatorname{Im} \partial_3$ has become trivial. Also $H_3(X_4) = \ker \partial_3 = 0$. 
The group $H_1(X_4)$ is the same as $H_1(X_3)$, namely $\mathbb{Z} \times \mathbb{Z}$, so this is the only nontrivial homology group of $X_4$. 

\paragraph{General case}
\begin{definition}
  For a cell complex $X$ the \emph{chain groups} $C_n(X)$ are the free abelian groups with basis the $n$-cells of $X$, with \emph{boundary homomorphisms} 
  $$\partial_n C_n(X) \to C_{n-1} (X)$$
  in terms of which one defines the \emph{homology group} $H_n(X) = \ker \partial_n / \operatorname{Im} \partial_{n+1}$. 
\end{definition}

The major difficulty is to define $\partial_n$ in general, for $n = 1$ this is simple, the boundary of an oriented edge is the vertex at the head 
minus the one at its tail. The case for $n = 2$ is also not hard, at least for cells attached along cycles that are loops of edges. But for large $n$, 
this becomes more complicated. 

The best solution is to subdivide polyhedra into special polyhedra called simplices (in $2$ or $3$ dimensions. these are the triangle and 
the tetrahedron) and then study these. There is no loss of generality, but some loss of efficiency by doing this. 
However, for simplices, there is no difficulty in defining boundary maps or in handling orientations. This is how we obtain \emph{simplical 
homology}. Still this doesn't solve all of our problems.

The way around this problem is to consider the collection of all possible continuous maps of simplices into a given space $X$. 
These maps generate tremendously large chain groups $C_n(X)$, but the quotients $H_n(X) = \ker \partial_n / \operatorname{Im} \partial_{n+1}$, 
called \emph{singluar homology groups}, turn out to be much smaller (for reasonnably nice spaces $X$). In particular, for space like those in the examples above, 
the singular homology groups are the same as the homology groups. This approach allows us to solve the problem 
of defining the boundary maps for cellular chains. 

\subsection{$\Delta$-complexes} 

 The torus, the projective plane, and the Klein bottle 
can each be obtained from a square by identifying opposite edges. 
Cutting a square along a diagonal produces two triangles, so each of these surfaces can also be built from two 
triangles by identifying their edges in pairs. Similarly, we decompose any polygon into triangles, 
in fact: all closed surfaces can be constructed from triangles by identifying edges. 

\begin{definition}
  The idea of a $\Delta$-complex is to generalize constructions like these to any number of dimensions. The $n$-dimensional 
  analogue of the triange is the \emph{$n$-simplex}: \\
  This is the smallest convex set in a Euclidian space $\mathbb{R}^m$ containing $n+1$ points $v_0, \cdots, v_n$ 
  that do not lie in a hyperplane of dimension less than $n$.

  Equivalently, we say that the difference vectors $v_1 - v_0, \cdots, v_n - v_0$ are L.I.
  The $v_i$ are the \emph{vertices} of the simplex, and the simplex itself is denoted $[v_0, \cdots, v_n]$. 
  Note that the odrering of the vertices is important here, since it gives an orientation ot the edges $[v_i, v_j]$ according 
  to increasing subscribt. 
\end{definition}

For example, the standard $n$-simplex is 
$$ \Delta^n = \{(t_0, \cdots, t_n) \in \mathbb{R}^{n+1} \mid \sum_i t_i = 1 \text{ and } t_i \geq 0 \ \forall i\}$$
whose vertices are the unit vectors along the coordinate axes. 

Notice that the ordering of the vertices determines a lienar homeomorphism from the standard $n$-simplex 
$\Delta^n$ to any other $n$-simplex $[v_0, \cdots, v_n]$ preserving the order of the vertices, namely 
$$(t_0, \cdots, t_n) \tp \sum_i t_i v_i$$

\begin{definition}
  If we delete one of the $n+1$ vertices of an $n$-simplex $[v_0, \cdots, v_n]$, then the remaining 
  $n$-vertices span an $(n-1)$-simplex, called a \emph{face} $[v_0, \cdots, v_n]$. 
\end{definition}

\begin{definition}
  THe \emph{boundary} of $\Delta^n$ is the union of all faces of $\Delta^n$, written $\partial \Delta^n$. 
  The open simplex $\mathring{\Delta}^n$ is $\Delta^n - \partial \Delta^n$, the interior of $\Delta^n$.
\end{definition}

A \emph{$\Delta$-complex} structure on a space $X$ is a collection of maps $\sigma_\alpha \colon \Delta^n \to X$, 
with $n$ depending on $\alpha$, such that: 
\begin{enumerate}[label = (\roman*)]
  \item The restriction $\sigma_\alpha | \mathring{\Delta^n}$ is injective, and each 
  point of $X$ is in the image of exactly one such restriction $\sigma_\alpha | \mathring{\Delta^n}$.

  \item Each restriction of $\sigma_\alpha$ to a face fo $\Delta^n$ is one of the maps 
  $$\sigma_\beta \colon \Delta^{n-1} \to X$$
  We are identifying the face of $\Delta^n$ with $\Delta^{n-1}$ by the canonical homeomorphism between 
  them that preserves the ordering of the vertices. 

  \item A set $A \subset X$ is open iff $\sigma_\alpha^{-1} (A)$ is open in $\Delta^n$ 
  for each $\sigma_\alpha$.
\end{enumerate}

THerefore we can build $X$ as a quotient space of a collection of disjoint simplices 
$\Delta^n_\alpha$, one for each $\sigma_\alpha \colon \Delta^n \to X$. One can think of 
building the quotient space inductively, starting with a discrete set of vertices, then 
attaching edges to these to produce a graph, then attaching $2$-simplices to the graph, and 
so no. Therefore a $\Delta$-complex can be described as collections of $n$-simplices $\Delta_\alpha^n$
for each $n$ together with attaching maps to associate faces of $n$ simplices with $n-1$ simplices. 

More generally, $\Delta$-complexes can be built from collections of disjoint simplices by identifying 
various subsimplices spanned by subsets of the vertices. 
The earlier $\Delta$-complex structures on a torus, projective plane, or Klein-bottle can be obtained 
in this way: by indetfying pairs of edges of two $2$-simplices. 
Note that sometimes we might be forced to subdivide our space into smaller $n$-simplices to be able to 
construct is as a $\Delta$-complex. 

\paragraph{Hausdorffness} Since a $\Delta$-complex $X$ is a quotient space of disjoint simplices, $X$ 
must be Hausdorff. Therefore by condition (iii) each restriction $\sigma_\alpha | \mathring{\Delta^n}$ 
is a homeomorphism onto its image, which is thus an open simplex in $X$. Therefore 
these open simplices $\simga_\alpha (\mathring{\Delta^n})$ are the cells $e_\alpha^n$ of a CW 
complex structure on $X$.

\subsection{Simplical Homology} 

The goal is to define simplical homology groups of a $\Delta$-complex $X$. Let 
$\Delta_n (X)$ be the free abelian group with basis the open $n$-simplices $e^n_\alpha$ of $X$. 
ELements of $\Delta_n (X)$, called $n$-chains, can be written as finite sums 
$$\sum_\alpha n_\alpha e^n_\alpha \quad n_\alpha \in \mathbb{N}$$
Equivalently, we could write $\sum_\alpha n_\alpha \sigma_\alpha$ where $\sigma_\alpha \colon \Delta^n \to X$ 
is the charactestic map of $e^n_\alpha$, with image the closure of $e^n_\alpha$. Such 
a sum can be thought of as a finite collection of $n$-simplices in $X$ with integer multiplicites, the coefficients 
$n_\alpha$. 

The boundary of the $n$-simplex $[v_0, \cdots, v_n]$ consists of the various  $(n-1)$-dimensional 
simplices $[v_0, \cdots, \hat{v_i}, \cdots, v_n]$ where the hat over $v_i$ indicates that this 
vertex is omitted. 
We might wish to say that the boundary of $[v_0, \cdots, \hat{v_i}, \cdots, v_n]$ is the $(n-1)$-chain 
$\sum_i [v_0, \cdots, \hat{v_i}, \cdots, v_n]$, the signs are inserted to take orientations into account. 

\begin{definition}
  With this geometry in mind, we define $\Delta$-complex $X$ a \emph{boundary homomorphism} 
  $$\partial_n \colon \: \Delta_n (X) \to \Delta_{n-1} (X)$$
  by specifying its values on basis elements 
  $$\partial_n (\sigma_\alpha) = \sum_i (-1)^i \sigma_\alpha | [v_0, \cdots, \hat{v_i}, \cdots, v_n]$$  
\end{definition}

\begin{lemma}
  The composition 
  $$\Delta_n (X) \xrightarrow{\partial_n} \Delta_{n-1}(X) \xrightarrow{\partial_{n-1}} \Delta_{n-2} (X)$$
  is zero, i.e. $\partial_{n-1} \circ \partial_n \colon \Delta_n \to \Delta_{n-2} = 0$ for all $n$.
\end{lemma}

The algebraic situation we have now is a sequence of homomorphisms of abelian groups 
$$\cdots \rightarrow C_{n+1} \xrightarrow{\partial_{n+1}} C_n \xrightarrow{\partial_n} C_{n-1} \rightarrow \cdots 
\rightarrow C_1 \xrightarrow{\partial_1} C_0 \xrightarrow{\partial_0} 0$$
with $\partial_n \partial_{n+1} = 0$ for all $n$. We call such a sequence a 
\emph{chain complex} (note that we have extended it by $0$ at the end with $\partial_0 = 0$). 
The equation $\partial_n \partial_{n+1} = 0$ can be rewritten as the inclusion 
$$\operatorname{Im} \partial_{n+1} \subset \ker \partial_n$$

\begin{definition}
  We define the \emph{$n^\operatorname{th}$ homology group} of the chain complex 
  to be the quotient group $H_n = \ker \partial_n / \operatorname{Im} \partial_{n+1}$,
\end{definition}

Elements of $\ker \partial_n$ are called \emph{$n$-cycles}, and elements of $\operatorname{Im} \partial_{n+1}$ are called 
\emph{$n$-boundaries}. Elements of $H_n$ are cosets of $\operatorname{Im} \partial_{n+1}$, 
called \emph{homology classes}. Two cycles representing the same homology class are said to be 
homologous, this means their difference is a boundary. 

\begin{definition}
  Returning to the case that $C_n = \Delta_n (X)$, the homology group $\ker \partial_n / \operatorname{Im} \partial_{n+1}$ will 
  be denoted $H_n^\Delta (X)$ and called the \emph{$n^\operatorname{th}$ simplical group} of $X$. 
\end{definition}

Some obvious questions arise: Are the homology groups independent of the 
choice of $\Delta$-complex structure on $X$? What about if they're homotopy equivalent? 
To answer these questions, it is better to leave the rigid simplical realm and 
introduce singular homology groups. 
These have the added advantage that they can be defined for all spaces (not just $\Delta$-complexes). 
However, for $\Delta$-complexes, the singular homology groups are isomorphic to the simplical homology groups.

Traditionally, simplical homology groups are defined for \emph{simplical complexes} which are
$\Delta$-complexes whose simplices are uniquely determined by their vertices. 
This ammounts to saying that each $n$-simplex has $n+1$ distcing vertices, and that 
no other $n$-simplex has this same set of vertices. Thus a simplical complex can be described 
as a set $X_0$ of vertices together with sets $X_n$ of $n$-simplices, 
which are $(n+1)$-element subsets of $X_0$. 
The only restriction is that each $(k+1)$-element subset of the vertices in $X_n$ are 
are the vertices of a $k$-simplex in $X_k$. From this combinatorial data, we 
can construct a $\Delta$-complex $X$ can be constructed by taking a partial ordering of the vertices 
$X_0$ that restricts to a linear ordering of each simplex in $X_n$.

It can be shown that every $\Delta$-complex can be subdivided to be a simplical complex. 
In particular, every $\Delta$-complex is homeomorphic to a simplical complex. 

Compared with simplical complexes, $\Delta$-complexes have the advantage of simpler computations since 
we need less simplices. 

\subsection{Singular Homology} 
\begin{definition}
  A \emph{singular $n$-simplex} in a space $X$ is defined to be a \textbf{continuous} map 
  $$\sigma \colon \: \Delta^n \to X$$
  The word 'singular' expresses the fact that $\sigma$ need not be a nice embedding, but can 
  have 'singularities' where the image does not look like a simplex. 
\end{definition}

Let $C_n (X)$ be the free abelian group with basis the set of singular $n$-simplices in $X$. 
Elements of $C_n (X)$, called \emph{singular $n$-chains}, are finite formal sums 
$\sum_i n_i \sigma_i$ for $n_i \in \mathbb{Z}$ and $\sigma_i \colon \Delta^n \to X$. 

A boundary map $\partial_n \colon C_n (X) \to C_{n-1} (X)$ is defined by 
$$\partial_n (\sigma) = \sum_i \left(-1\right)^i \sigma | [v_0, \cdots, \hat{v_i}, \cdots, v_n]$$
Where we identify $[v_0, \cdots, \hat{v_i}, \cdots, v_n]$ with $\Delta^{n-1}$, preserving the order of the 
vertices so that $\sigma | [v_0, \cdots, \hat{v_i}, \cdots, v_n]$ is a map $\Delta^{n-1} \to X$, i.e a 
singular $(n-1)$-simplex. 

\begin{lemma}
  Similarly as above, the composition of boundaries is null, i.e 
  $$\partial_n \parital_{n+1} = 0$$
  Or more concisely $\partial^2 = 0$. 
\end{lemma}

\begin{definition}
  We define the \emph{singular homology group} 
  $$H_n (X) = \ker \partial_n / \operatorname{Im} \partial_{n+1}$$
  Note that this is well defined by the lemma above. 
\end{definition}

Notice that homeomorphic spaces have isomorphic singular homology grups $H_n$, in contrast 
with the situation for $H_n^\Delta$. On the other hand, since the $C_n (X)$ are so large, 
it is not clear that for a $\Delta$-complex $X$ with finitely many simplices $H_n(X)$ should be 
finitely generated for all $n$, or that $H_n(X)$ should be zero for $n$ larger than the dimension of $X$;
two properties which are trivial for $H_n^\Delta (X)$. 

Surpisingly, singular homology can be regarded as a special case of simplical homology by means of the following 
construction: 
\begin{definition}
  \item For an arbitary space $X$, define the \emph{singular complex} $S(X)$ to be the $\Delta$-complex
  with one $n$-simplex $\Delta^n_\sigma$ for each singular $n$-simplex $\sigma \colon \Delta^n \to X$. 
  with $\Delta_\sigma^n$ attached in the obvious way to the $(n-1)$-simplices of $S(X)$ 
  that are the restrictions of $\sigma$ to the various $(n-1)$-simplices of $\partial \Delta^n$.
\end{definition}
It is clear from this deifnition that $H_n^\Delta (X)$ is identical with $H_n(X)$ for all $n$, and in this sense 
the singular homology groups are a special case of a simplical homology group. 
Once can regard $S(X)$ as a $\Delta$-complex model for $X$, although it is usually much larget than $X$.

\paragraph{Geometric picture of homology classes} 
Although we defined cycles algebraically, we can give them a geometric interpretation in terms 
of maps from finite $\Delta$-complexes into $X$. To see this, note that a singular $n$-chain $\xi$ can always 
be written in the form $\sum_i \varepsilon_i \sigma_i$ with $\varepsilon_i = \pm 1$, allowing repetitions of the 
$\sigma_i$. Given such an $n$-chain $\xi = \sum_i \varepsilon_i \sigma_i$, 
when we compute $\partial \xi$ as a sum of singular $(n-1)$-simplices with signs $\pm 1$, 
there may be some \emph{canceling pairs} consisting of two identical singular $(n-1)$-simplices with opposite 
signs. 
Choosing a maximal collection of such canceling pairs, construct an $n$-dimensional $\Delta$-complex 
$K_\xi$ from a disjoint union of $n$-simplices $\Delta_i^n$, one for each $\sigma_i$, 
by identifying the pairs of $(n-1)$-simplices that cancel each other. The $\sigma_i$'s then induce a map 
$K_\xi \to X$. If $\xi$ is a cycle, all the $(n-1)$-dimensional faces of the $\Delta_i^n$ are identified in pairs 
Thus $K_\xi$ is a manifold (since it has no boundary), locally homeomorphic to $\mathbb{R}^n$, near all points in the complement 
of the $(n-2)$-skeleton of $K_\xi$. More precisely: $K_\xi \setminus K^{n-2}_\xi$ is an oriented 
$n$-manifold. 

In particular, homology classes can be realized by maps from $n$-pseudomanifolds: 
\begin{itemize} 
  \item $H_1(X)$ is generated by images of loops $S_1 \to X$ 
  \item $H_2 (X)$ is generated by images of closed oriented surfaces $\Sigma \to X$. 
\end{itemize}

Moreover, a loop is null in $H_1(X)$ iff it bounds some compact oriented surface in $X$. Similarly
a closed surface is null in $H_2(X)$ iff it extends to a map of some compact oriented $3$-manifold with that surface as its boundary.

\begin{proposition}
  Let $X_\alpha$ be the decomposition of $X$ into its path-connected components, then 
  $$H_n(X) \cong \bigoplus_\alpha H_n(X_\alpha)$$
\end{proposition}

\begin{proposition}
  If $X$ is nonempty and path-connected, then $H_0(X) \cong \mathbb{Z}$. 
  Hence for any $X$ we get 
  $$H_0(X) = Z^\alpha$$
  where $\{X_\alpha\}$ is the decomposition of $X$ into its path-connected components.
\end{proposition}

\begin{proposition}
  If $X$ is a point, then $H_n(X) = 0$ for all $n > 0$, and $H_0(X) \cong \mathbb{Z}$. 
\end{proposition}

\begin{definition}
  The \emph{reduced homology groups} $\tilde{H}_n (X)$ are the homology groups of the augmented chain complex 
  $$\cdots \rightarrow C_2(X) \xrightarrow{\partial_2} C_1 (X) \xrightarrow{\partial_1} C_0 (X) \xrightarrow{\varepsilon} \mathbb{Z} \rightarrow 0$$
  Where $\varepsilon \left(\sum_i n_i \sigma_i\right) = \sum_i n_i$. 
\end{definition}
This allows to define a homology for which a single point has trivial homology even for $n = 0$. 

Since $\varepsilon \partial_1 = 0$, $\varepsilon$ induces a map $H_0(X) \to \mathbb{Z}$ with kernel 
$\tilde{H}_0(X)$ so $H_0(X) \cong \mathbb{Z} \oplus \tilde{H}_0(X)$. 

Formally one can think of the extra $\mathbb{Z}$ as generated by the unique map $\emptyset \to X$, then $\varepsilon$ becomes the usual boundary 
map since $\partial[v_0] = [\hat{v_0}] = \emptyset$. 

\subsection{Relative Homology} 

If there was a simple relationship between the homology groups of a space $X$, a subspace $A$, 
and the quotient $X / A$, this would be a useful tool in understanding the homology groups 
of spaces such as CW complexes. 
Ideally, we would have $H_n(A) \subset H_n(X)$ with $H_n(X) / H_n(A) \cong H_n(X/A)$.
While this holds in some cases, if it held in general this would collapse homology, since 
every $X$ can be embedded as a subspace of a space with trivial homology groups, 
namely the cone $CX = (X \times I) / X \times \{0\}$ which is contractible. 

The modification that we make to the relationship is that it should involve $H_n(X), H_n (A), H_n(X/A)$ for 
all values of $n$ simultaneously. This also allows us to compute higher-dimensional 
groups in terms of lower-dimensional ones. 

\begin{definition}
  A sequence of homomorphisms 
  $$ \cdots \rightarrow A_{n+1} \xrightarrow{\alpha_{n+1}} A_n \xrightarrow{\alpha_n} A_{n-1} \rightarrow \cdots$$
  is said to be \emph{exact} if $\Ker \alpha_n = \operatorname{Im} \alpha_{n+1}$ for each $n$. 
\end{definition}
Notice that $\operatorname{Im} \alpha_{n+1} \subset \ker \alpha_n$ is equivalent to 
$\alpha_n \alpha_{n+1} = 0$, so the sequence is a chain complex. 
Moreover the inclusion $\ker \alpha_n \subset \operatorname{Im} \alpha_{n+1}$ say that the 
homology groups $H_n$ are trivial. 

This definition is very useful, and allows us to express a number of algebraic concepts: 
\begin{enumerate}[label = (\roman*)]
  \item $0 \to A \xrightarrow{\alpha} B$ is exact iff $\ker \alpha = 0$, i.e $\alpha$ is injective. 
  \item $A \xrightarrow{\alpha} B \to 0$ is exact iff $\operatorname{Im} \alpha = B$, i.e $\alpha$ is surjective. 
  \item $0 \to A \xrightarrow{\alpha} B \to C$ is exact iff $\alpha$ is an isomorphism. 
  \item $0 \to A \xrightarrow{\alpha} B \xrightarrow{\beta} C \to 0$ is exact iff $\alpha$ is injective, 
  $\beta$ is surjective, and $\ker \beta = \operatorname{Im} \alpha$, so $\beta$ induces an isomorphism 
  $C \cong B / \operatorname{Im} \alpha$. This can be written as $C \cong B / A$ if we look at 
  $\alpha$ as an inclusion $A \subset B$. 
\end{enumerate}

An exact sequence $0 \to A \to B \to C \to 0$ is called a \emph{short exact sequence}. 

\begin{theorem}
  If $X$ is a space and $A$ is a nonempty closed subspace of $X$ that 
  is a deformation retract of a neighborhood in $X$, then there is an exact sequence
  $$\cdots \to \tilde{H}_n(A) \xrightarrow{i_*} \tilde{H}_n(X) \xrightarrow{j_*}
  \tilde{H}_n (X / A) \xrightarrow{\partial} \tilde{H}_{n-1}(A) \xrightarrow{i_*} H_{n-1}(X) \to \cdots \to \tilde{H}_0(X/A) \to 0$$
  Where $i$ is the inclusion $A \hookrightarrow X$ and $j$ is the quotient map $X \to X/A$. 
\end{theorem}
An element $x \in \tilde{H}_n(X/A)$ can be represented by a chain $\alpha$ in $X$ with $\partial \alpha$ a cycle in 
$A$ whose homology class is $\partial x \in \tilde{H}_{n-1}(A)$. 

\begin{definition}
  A pair of spaces $(X,A)$ is said to be \emph{good} if it rsepects the hypothesis of the theorem.
\end{definition}

\begin{corollary}
  $\tilde{H}_n (S^n) \cong \mathbb{Z}$ and $\tilde{H}_k (S^n) = 0$ for $k \neq n$. 
\end{corollary}

\begin{corollary}
  $\partial D^n$ is not a retract of $D^n$. Hence every map $f \colon D^n \to D^n$ has a fixed 
  point. 
\end{corollary}

\paragraph{Relative Homology groups} 
In relative homology, we ignore all singular chains in a subspace of the 
given chain. 

\begin{definition}
  Given a space $X$ and a subspace $A \subset X$, let $C_n(X,A)$ be the quotient group 
  $C_n(X) / C_n(A)$. Thus chains in $A$ are trivial in $C_n(X,A)$. 
  Since $\partial \colon \: C_n(X) \to C_{n-1}$ takes $C_n (A)$ to $C_{n-1} (A)$, 
  it induces a quotient boundary map $\partial \colon C_{n} (X, A) \to C_{n-1} (X,A)$. 
  Letting $n$ vary, we have a sequence of boundary maps 
  $$ \cdots \to C_n (X, A) \xrightarrow{\partial} C_{n-1} (X,A) \to \cdots$$
  Clearly, the relation $\partial^2 = 0$ holds for there boundary maps. So we have a 
  chain complex, and the homology groups $\ker \partial / \operatorname{Im} \partial$ of this 
  chain  complex are by definition \emph{relative homology groups} $H_n(X,A)$. 
\end{definition}

By definition of the relative boundary map, we notice: 
\begin{itemize}
  \item Elements of $H_n(X,A)$ are represented by \emph{relative cycles}: $n$-chains $\alpha \in C_n(X)$ such that 
  $\partial \alpha \in C_{n-1} (A)$. 
  \item A relative cycle $\alpha$ is trivial in $H_n(X,A)$ iff it is a 
  \emph{relative boundary}: $\alpha = \partial \beta + \gamma$ for some $\beta \in C_{n+1} (X)$ and $\gamma \in C_n (A)$. 
\end{itemize}
Therefore intuitively, $H_n(X,A)$ is the 'homology of $X$ modulo $A$'. 

The quotient $C_n(X) / C_n(A)$ could also be viewed as a subgroup of $C_n(X)$, the free abelian subgroup with 
basis the singular $n$-simplices $\sigma \colon \Delta^n \to X$ whose image is not contained in $A$. 
However, the boundary map does not take this subgroup of $C_n(X)$ to the corresponding subgroup of $C_{n-1} (X)$, so it 
is better to regard $C_n (X, A)$ as a quotient rather than a subgroup of $C_n(X)$. 

Our goal now it to show that the relative homology groups $H_n(X,A)$ for any pair fits into the long exact sequence 
$$\cdots \to H_n (A) \to H_n(X) \to H_{n-1} (A) \to H_{n-1} (X) \to \cdots \to H_0(X,A) \to 0$$
We start with the diagram 
\[
\begin{tikzcd}
0 \arrow[r] &
C_n(A) \arrow[r,"i"] \arrow[d,"\partial"] &
C_n(X) \arrow[r,"j"] \arrow[d,"\partial"] &
C_n(X,A) \arrow[r] \arrow[d,"\partial"] &
0 \\
0 \arrow[r] &
C_{n-1}(A) \arrow[r,"i"] &
C_{n-1}(X) \arrow[r,"j"] &
C_{n-1}(X,A) \arrow[r] &
0
\end{tikzcd}
\]
Where $i$ is inclusion and $j$ is quotient, the diagram is clearly commutative. 
Letting $n$ vary, and drawing these short exact sequences vertically, we get the following diagram.
\[
\begin{tikzcd}
& 0 \arrow[d] & 0 \arrow[d] & 0 \arrow[d] & \\
\cdots \arrow[r] &
A_{n+1} \arrow[r,"\partial"] \arrow[d,"i"] &
A_n \arrow[r,"\partial"] \arrow[d,"i"] &
A_{n-1} \arrow[r,"\partial"] \arrow[d,"i"] &
\cdots \\
\cdots \arrow[r] &
B_{n+1} \arrow[r,"\partial"] \arrow[d,"j"] &
B_n \arrow[r,"\partial"] \arrow[d,"j"] &
B_{n-1} \arrow[r,"\partial"] \arrow[d,"j"] &
\cdots \\
\cdots \arrow[r] &
C_{n+1} \arrow[r,"\partial"] \arrow[d] &
C_n \arrow[r,"\partial"] \arrow[d] &
C_{n-1} \arrow[r,"\partial"] \arrow[d] &
\cdots \\
& 0 & 0 & 0 &
\end{tikzcd}
\]
\[
\text{with } A_*=C_*(A),\quad B_*=C_*(X),\quad C_*=C_*(X,A),
\]

Such a diagram is called a \emph{short exact sequence of chain complexes}. We will show that this 
short exact sequence of chain complexes streches out into a long exact sequence of homology groups 
$$\cdots H_n(A) \xrightarrow{i_*} H_n(B) \xrightarrow{j_*} H_n (C) \xrightarrow{\partial} H_{n-1} (A) \xrightarrow{i_*} H_{n-1} B \rightarrow \cdots$$
Where $H_n(A)$ denotes the homology group $\ker \partial / \operatorname{Im} \partial$ at $A_n$ in the chain complex $A$, 
similarly for $H_n(B)$ and $H_n(C)$. 

Thanks to the commutativity of the diagram, we know that $i$ and $j$ are chain maps, therefore 
they induce maps $i_*$ and $j_*$ on homology. We still need to define the boundary map 
$\partial \colon H_n(C) \to H_{n-1} (A)$, let $c \in C_n$ be a cycle and let $b \in B_n$ such that 
$c = j(b)$. Notice that 
$$j(\partial b) = \partial j(b) = \partial c = 0 \quad \text{$c$ is a cycle}$$
so $\partial b \in \ker j$, with $\ker j = \operatorname{Im} i$, therefore 
$$\partial b = i(a) \quad a \in A_{n-1}$$
But note that 
$$i(\partial a) = \partial i(a) = \partial ^2 b = 0$$
so by injectivity of $a$ we get $\partial a = 0$. This allows us to finally define 
$\partial \colon H_n (C) \to H_{n-1} (A)$ by 
$$\partial [c] = [a]$$
Note that this is well defined, since: 
\begin{itemize}
  \item The element $a$ is uniquely determined by $\partial b$ since $i$ is injective. 
  \item A different choice $b'$ would have $j(b') = j(b)$ so $b' - b \in \ker j = \operatorname{Im} i$. 
  Thus $b' - b = i(a')$, hence $b' = b + i(a')$. Therefore we are chaning $a$ to the homologous element 
  $a + \partial a'$ since $i (a + \partial a')  = \partial (b + i(a'))$
  \item A different choice of $c$ within its homology class would have the form $c + \partial c'$. 
  Since $c' = j(b')$, we get $c + \partial c' = c + \partial j(b') = j(b + \partial b')$, 
  so $b$ is replaced by $b + \partial b'$, which leaves $\partial b$ and there $a$ unchanged. 
\end{itemize}
The map $\partial \colon H_n (C) \to H_{n-1} (A)$ is a homomorphism since if $\partial [c_1] = [a_1]$ and 
$\partial [c_2] = [a_2]$ via elements $b_1$ and $b_2$ as above, then $j(b_1 + b_2) = j(b_1) + b_2 = c_1 + c_2$ 
and $i(a_1 + a_2) = i(a_1) + i(a_2) = \partial (b_1 + b_2)$, so $\partial ([c_1] + [c_2]) = [a_1] + [a_2]$. 

\begin{theorem}
  The sequence of homology groups 
  $$\cdots \rightarrow H_n(A) \xrightarrow{i_*} H_n(B) \xrightarrow{j_*} H_n (C) \xrightarrow{\partial} H_{n-1} (A) \xrightarrow{i_*} H_{n-1} (B) \rightarrow \cdots$$
  is exact. 
\end{theorem}

Returning to toplogy, the preceeding algebraic theorem yields a long exact sequence of homology groups: 
$$\cdots \rightarrow H_n(A) \xrightarrow{i_*} H_n(X) \xrightarrow{j_*} H_n (X,A) \xrightarrow{\partial} H_{n-1} (A) \xrightarrow{i_*} H_{n-1} (X) \to \cdots \to H_0(X,A) \to 0$$
The boundary map $\partial \colon H_n(X,A) \to H_{n-1} (A)$ has a very simple description: 
If a class $[\alpha] \in H_n(X,A)$ is represented by a cycle $\alpha$, then $\partial [\alpha]$ is the class of the cycle $\partial \alpha$ in 
$H_{n-1} (A)$. 

This makes precise the idea that the groups $H_n(X,A)$ measures the difference between the groups $H_n(X)$ and 
$H_n(A)$. In particular, exactness implices that if $H_n(X,A) = 0$ for all $n$, then the inclusion $A \hookrightarrow X$ induces 
isomorphisms $H_n(A) \cong H_n(X)$ for all $n$, the converse is also true. 

There is a completely analogous long exact sequence of reduced homology groups for a pair $(X,A)$ with $A \neq \emptyset$. 
This comes from applying the preceeding algebraic machinery to the short exact sequences 
$0 \to C_n(A) \to C_n(X) \to C_n(X,A) \to 0$ in nonnegative dimensions, augmented by the short exact sequence 
$0 \to \mathbb{Z} \xrightarrow{\mathbb{1}} \mathbb{Z} \to 0 \to 0$ in dimension $-1$. In particular this means that 
$\tilde{H}_n(X,A) \cong H_n (X,A)$ for all $n$ when $A \neq \emptyset$. 

There are induced homomorphisms for relative homology just as there are in the nonrelative case. A map $f \colon X \to Y$ with 
$f(A) \subset B$ (or more concisely a map $f_\# \colon (X, A) \to (Y, B)$) induces homomorphisms 
$$f_\# \colon C_n(X,A) \to C_n(Y,B)$$ 
Since the chain maps $f_\# \colon C_n(X) \to C_n(Y)$ sends $C_n(A) \to C_n (B)$, we get a well defined map on quotients 
$$f_\# \colon C_n(X,A) \to C_n(Y,B)$$
And since the relation $f_\# \partial = \partial f_\#$ holds for absolute chains, it holds for relative chains and therefore we 
have induced homomorphisms on homology 
$$f_* \colon H_n(X, A) \to H_n(Y,B)$$

\paragraph{Excision} 
A fundamental property of relative homology groups is given by the following \emph{excision theorem}, describing when the relative groups 
$H_n(X,A)$ are unaffected by deletign (or excising) a subset $Z \subset A$. 

\begin{theorem}
  Given subspaces $Z \subset A \subset X$ such that the closure of $Z$ is contained in the interior of $A$, 
  then the inclusion $(X- Z, A - Z) \hookrightarrow (X, A)$ induces isomorphisms 
  $$H_n(X - Z, A - Z) \cong H_n(X,A) \quad \forall n$$
  Equivalently, for subspaces $A, B \subset X$ whose interior cover $X$, the inclusion 
  $(B, A \cap B) \hookrightarrow (X, A)$ induces isomorphisms 
  $$H_n (B, A \cap B) \cong H_n(X, A) \quad \forall n$$
\end{theorem}

The translation between the two versions is obtained by setting $B = X - Z$ and $Z = X - B$. Then $A \cap B = A - Z$ 
and the condition $\operatorname{cl} Z \subset \operatorname{int} A$ is equivalent to 
$X = \operatorname{int} A \cup \operatorname{int} B$. 


\section{References} 
\begin{quote}
\textbf{Tammo Dieck's} Book \emph{Algebraic Topology} 

\textbf{Allen Hatcher's} Book \emph{Algebraic Topology} 

\textbf{Martin Furter's} Playlist \emph{Topology} 

\textbf{Math at Andrews University} Playlist \emph{Algebraic topology}
\end{quote}
\end{document}